\documentclass[svgnames,hyperref={bookmarks=false}]{beamer}
\usepackage[utf8]{inputenc}
\usepackage[english,russian]{babel}
\useoutertheme{infolines}
\setbeamertemplate{headline}{} % removes the headline that infolines inserts
% \setbeamertemplate{footline}{
%   \hfill%
%   \usebeamercolor[fg]{page number in head/foot}%
%   \usebeamerfont{page number in head/foot}%
%   \insertpagenumber\,/\,\insertpresentationendpage\kern1em\vskip2pt%
% }
\setbeamertemplate{footline}{
  \hfill%
  \usebeamercolor[fg]{page number in head/foot}%
  \usebeamerfont{page number in head/foot}%
  \footnotesize\insertpagenumber\kern1em\vskip2pt%
}
\setbeamertemplate{navigation symbols}{}
\setbeamercolor{alerted text}{fg=blue}
\setbeamerfont{alerted text}{series=\bfseries,family=\ttfamily}
\setbeamertemplate{section in toc}{\text{\color{black}{\inserttocsection}}}
% \setbeamercolor{titlelike}{bg=white,fg=black}
\usepackage[parfill]{parskip}
\usepackage[linewidth=0.5pt]{mdframed}
\newmdenv[innerleftmargin=1mm, innerrightmargin=1mm, innertopmargin=-1mm, innerbottommargin=2mm, leftmargin=-1mm, rightmargin=-1mm]{lstlistinglike}
\usepackage{tikz}
\usepackage{graphicx}
\DeclareGraphicsExtensions{.pdf,.png,.jpg}
\usepackage{textcomp}
\usepackage{pifont}
\usepackage{textpos}

\hypersetup{pdfauthor={Евгений Бурмако},pdfsubject={Макросы в Скале},pdftitle={Макросы в Скале}}
\title{Макросы в Скале}
\titlegraphic{\includegraphics[width=4cm]{epfl.png}}

\begin{document}

\title{Макросы в Скале}
\author{Евгений Бурмако}
\institute{\'Ecole Polytechnique F\'ed\'erale de Lausanne \\
           \texttt{http://scalamacros.org/}}
\date{31 марта 2013}
{
\setbeamertemplate{footline}{}
\begin{frame}
  \titlepage
\end{frame}
}

\begin{frame}[fragile]
\frametitle{О чем речь?}

Доклад в двух предложениях:
\begin{itemize}
\item Зачем люди используют кодогенерацию?
\item Что в этом плане можно сделать в Скале?
\end{itemize}
\end{frame}

\begin{frame}[fragile]
\frametitle{Копипаста}

Копипаста это плохо.
\end{frame}

\begin{frame}[fragile]
\frametitle{Абстракция}

\begin{itemize}
\item Все мы умеем абстрагироваться от частностей и обобщать повторяющийся код
\item И все мы знаем, что иногда это не работает
\end{itemize}
\end{frame}

\begin{frame}[fragile]
\frametitle{Примеры}

\begin{itemize}
\item Геттеры-сеттеры-конструкторы
\item Публикация и подписка на события
\item Переносчики данных
\item Доменно-специфические языки
\end{itemize}
\end{frame}

\begin{frame}[fragile]
\frametitle{Что делать?}

\begin{itemize}
\item Мучительно писать горы однотипного кода
\item Колбасить кодогенераторы
\item Поменять язык программирования
\only<2>{
  \begin{itemize}
    \item Но нет предела совершенству
    \item Даже на Хаскелле есть пятна
    % TODO: собрать воедино аргументацию на основе дискашенов из ЖЖ
  \end{itemize}
}
\end{itemize}
\end{frame}

\begin{frame}[fragile]
\frametitle{Кодогенерация}

\begin{itemize}
\item Плагины к билд тулу, генерирующие текст
\item Текстовые макросы (C/C++)
\item Синтаксические макросы (Lisp, Nemerle, Haskell, Scala)
\item Постпроцессоры байткода
\item Рантаймовая кодогенерация
\end{itemize}
\end{frame}

\begin{frame}[fragile]
\frametitle{}

\vskip40pt
\begin{center}
\text{\color{blue}{\Large{Макросы в Скале 2.10.0}}}
\end{center}
\end{frame}

\begin{frame}[fragile]
\frametitle{Что умеют?}

Превращать вызовы специально объявленных функций в произвольный код.

\vskip20pt
Было:
\begin{semiverbatim}
assert(2 + 2 == 4, "прикольное сообщение об ошибке")
\end{semiverbatim}

\vskip20pt
Стало:
\begin{semiverbatim}
if (2 + 2 != 4) raise("прикольное сообщение об ошибке")
\end{semiverbatim}
\end{frame}

\begin{frame}[fragile]
\frametitle{Как это работает?}

\begin{itemize}
\item Макросы = функции, которые вызывает компилятор
\item В эти функции передается код программы
\item Эти функции возвращают код
\item Сгенерированный код компилируется как написанный вручную
\item Детали: \text{\color{blue}{\href{http://scalamacros.org/documentation.html}{http://scalamacros.org/documentation.html}}}
\end{itemize}
\end{frame}

\begin{frame}[fragile]
\frametitle{Пример}
\begin{semiverbatim}
def assert(cond: Boolean, msg: Any) = macro impl
def impl(c: Context)(cond: c.Expr[Boolean], msg: c.Expr[Any]) = \{
  if (assertionsEnabled)
    // \only<1>{if (!cond) raise(msg)}\only<2>{c.reify(if (cond.splice) raise(msg.splice))}\only<3>{q"if (\$cond) raise(\$msg)"}
    If(
      Select(cond.tree, newTermName("\$unary_bang")),
      Apply(Ident(newTermName("raise")), List(msg.tree)),
      Literal(Constant(())))
  else
    Literal(Constant(())
\}

import assert
assert(2 + 2 == 4, "прикольное сообщение об ошибке")
\end{semiverbatim}
\end{frame}

\begin{frame}[fragile]
\frametitle{Какие есть возможности?}

\begin{itemize}
\item Исполнение произвольного кода (макросы - это обычные Скальные функции)
\item Генерация произвольного кода (вызовы методов, объявления локальных классов/функций)
\item Интроспекция (код компилируемой программы, список мемберов классов, загрузка аннотаций)
\item Общение с тайпчекером (тайпчек, поиск имплиситов, проверка сабтайпинга)
\item Реификация кода
\item Сообщения об ошибках
\end{itemize}
\end{frame}

\begin{frame}[fragile]
\frametitle{Практические сложности с первой версией}

\begin{itemize}
\item Зачастую приходится работать на самом низком уровне (с синтаксическими деревьями)
\item API для макросов пока что неполный и не окончательный
\item Не получится генерировать классы, видимые извне раскрытия макроса
\item Для написания нетривиальных макросов надо знать как работает компилятор изнутри
\item Трудности с инфраструктурой (кэши, IDE и.т.д)
\item Сгенерированный код невозможно дебагать в интерактивном режиме
\end{itemize}
\end{frame}

\begin{frame}[fragile]
\frametitle{}

\vskip40pt
\begin{center}
\text{\color{blue}{\Large{Кто уже использует макросы}}}
\end{center}
\end{frame}

\begin{frame}[fragile]
\frametitle{Slick}

\begin{semiverbatim}
val coffees = Queryable[Coffee]
val result = coffees
  .filter(с => c.supId == 101)
  .map(c => (c.name, c.price))
\end{semiverbatim}

\begin{itemize}
\item Вроде бы обычные filter и map на самом деле являются макросами
\item Эти макросы ничего не делают с аргументами во время компиляции, а просто сохраняют эти аргументы на будущее (реификация)
\item В рантайме сохраненные аргументы (например, предикат фильтра) транслируются в запросы к базе данных
\end{itemize}
\end{frame}

\begin{frame}[fragile]
\frametitle{Play}

Скала 2.9:
\begin{semiverbatim}
case class Person(name: String, age: Int)
implicit val personReads = (
  (__ \ 'name).reads[String] and
  (__ \ 'age).reads[Int]
)(Person)
\end{semiverbatim}

Скала 2.10:
\begin{semiverbatim}
case class Person(name: String, age: Int)
implicit val personReads = Json.reads[Person]
\end{semiverbatim}

Скала 2.11:
\begin{semiverbatim}
case class Person(name: String, age: Int)
\end{semiverbatim}
\end{frame}

\begin{frame}[fragile]
\frametitle{SBT}

Было:
\begin{semiverbatim}
myTask <<= (aTask, bTask, cTask) map \{ (a,b,c) =>
  f(a, b, c)
\}
\end{semiverbatim}

Стало:
\begin{semiverbatim}
myTask := f(aTask.value, bTask.value, cTask.value)
\end{semiverbatim}

\begin{itemize}
\item Аналогичным способом можно создавать свои нотации и в других случаях, перегружая
разнообразные языковые конструкции (точку с запятой, условия, циклы, объявления и.т.д).
\end{itemize}

\end{frame}

\begin{frame}[fragile]
\frametitle{Sqltyped}

\begin{semiverbatim}
scala> val q = sql("select name, age from person")
scala> q() map (_ get age)
res0: List[Int] = List(36, 14)

scala> q() map (_ get salary)
<console>:24: error: No such key Columns.salary.type
               q() map (_ get salary)
\end{semiverbatim}

\begin{itemize}
\item Сиквел-запросы, валидируемые и типизируемые во время компиляции
\item Да, макросы натурально берут и стучатся в базу данных в компайл-тайме
\end{itemize}
\end{frame}

\begin{frame}[fragile]
\frametitle{Declosurify}

\begin{semiverbatim}
for (i <- 0 until 100000000 optimized) { ... }
\end{semiverbatim}

\begin{itemize}
\item Кодировать циклы на функциях высшего порядка это прикольно
\item Но Скала пока что не умеет эффективно компилировать такие функции, зачастую создавая промежуточные структуры данных, которые тормозят исполнение
\item При помощи макроса можно взять красиво записанный цикл и преобразовать его в некрасивую, но быструю форму - бенчмарки сыты, глаза целы
\end{itemize}
\end{frame}

\begin{frame}[fragile]
\frametitle{}

\vskip40pt
\begin{center}
\text{\color{blue}{\Large{Макросы в будущей Скале}}}
\end{center}
\end{frame}

\begin{frame}[fragile]
\frametitle{Прогресс не стоит на месте}
С момента кодофриза Скалы 2.10 мы работали над новыми фишечками, на покладая рук.
\end{frame}

\begin{frame}[fragile]
\frametitle{Что нас интересует?}
\begin{itemize}
\item Как сделать написание макросов гуманным?
\item Какие еще возможности дать нашим пользователям?
\item Как скрестить макросы и типы?
\end{itemize}
\end{frame}

\begin{frame}[fragile]
\frametitle{Квазицитаты}

Было:
\begin{itemize}
\item Apply(Ident(newTermName("future")), List(body))
\item AppliedTypeTree(Ident(newTermName("Future")), List(tpt))
\item case ClassDef(\_, name, \_, Template(\_, \_, \_ :: defs)) $\Rightarrow$ ...
\end{itemize}

\vskip25pt
Стало:
\begin{itemize}
\item q"future \{ \$body \}"
\item tq"Future[\$tpt]"
\item case q"class \$name \{ ..\$\{\_ :: defs\} \}" $\Rightarrow$ ...
\end{itemize}
\end{frame}

\begin{frame}[fragile]
\frametitle{Тайп макросы}

\begin{semiverbatim}
type H2Db(url: String) = macro impl

object Db extends H2Db("coffees")

val brazilian = Db.Coffees.insert("Brazilian", 99, 0)
Db.Coffees.update(brazilian.copy(price = 10))
println(Db.Coffees.all)
\end{semiverbatim}

\begin{itemize}
\item Уже как несколько месяцев макросы умеют генерировать публично видимые определения (классы, трейты, объекты)
\item Очевидное применение - избавление от переносчиков данных
\item Но есть еще и всякие трюки вроде более адекватных энумов
\end{itemize}
\end{frame}

\begin{frame}[fragile]
\frametitle{Макро аннотации}

\begin{semiverbatim}
@Serializable
class Foo(val x: Int, val y: String)
\end{semiverbatim}

\begin{itemize}
\item Совсем недавно я научился добавлять мемберы в произвольные классы в компилируемой программе.
\item Открывающиеся возможности очень широки, но пока до конца неясны
\item Как минимум:
\begin{itemize}
\item Кастомные кейс классы
\item Удобная сериализация
\item Более удобное реактивное программирование
\end{itemize}
\end{itemize}
\end{frame}

\begin{frame}[fragile]
\frametitle{Эзотерика}

Разнообразны взаимодействия макросов с системой типов Скалы:
\begin{itemize}
\item Имплисит макросы
\item Нетипизированные макросы
\item Макросы, влияющие на вывод типов
\item Вычисления на типах при помощи макросов
\end{itemize}
\end{frame}

\begin{frame}[fragile]
\frametitle{Macro paradise}

Прикольные штучки из этого раздела доклада пока что тусуются в стороне от официальной Скалы,
но некоторые из этих новых фич уже заинтересовали наших продакшен ребят.

Находятся штучки в моем репозитории, из которого собираются найтли билды.
Если погуглить по словам \texttt{macro paradise}, то найдутся инструкции по установке и использованию.
\end{frame}

\begin{frame}[fragile]
\frametitle{}

\vskip40pt
\begin{center}
\text{\color{blue}{\Large{Заключение}}}
\end{center}
\end{frame}

\begin{frame}[fragile]
\frametitle{Заключение}
\begin{itemize}
\item Макросы - плагины к компилятору с человеческим лицом
\item Они уже с нами в продакшен релизе Скалы 2.10.0
\item Такие популярные библиотеки как Slick, SBT и Play уже используют макросы
\item Мы не стоим на месте и постоянно развиваем концепцию макросов в Скале
\end{itemize}
\end{frame}

\begin{frame}[fragile]
\frametitle{}

\vskip40pt
\begin{center}
\text{\Large{Вопросы и ответы}}\\
\text{\color{blue}{\texttt{eugene.burmako@epfl.ch}}}
\end{center}
\end{frame}

\end{document}