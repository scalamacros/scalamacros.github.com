\documentclass[svgnames,dvipsnames,hyperref={bookmarks=false},usepdftitle=false]{beamer}
\usetheme{scaladays}
\usepackage{tikz}
\usetikzlibrary[arrows.meta]

\title{State of the Meta, Summer 2015}
\author{Eugene Burmako (\href{https://twitter.com/xeno_by}{@xeno{\textunderscore}by})}
\institute{\'Ecole Polytechnique F\'ed\'erale de Lausanne \\ \texttt{\href{http://scalameta.org/}{http://scalameta.org/}}}
\date{Presented on 09 June 2015\\ Last updated on 21 Sep 2015\\ Video available in \text{\color{blue}\href{https://www.parleys.com/tutorial/state-meta-summer-2015}{the ScalaDays collection}}}
\hypersetup{pdfauthor={Eugene Burmako},pdftitle={State of the Meta, Summer 2015}}

\begin{document}

\titleframe

\begin{frame}{scala.meta}
\begin{quote}
\small{Simple, robust and portable metaprogramming foundation for Scala}
\end{quote}
\begin{flushright}
\textemdash\text{ }\small{github.com/scalameta}
\end{flushright}
\end{frame}

\begin{frame}{Main goal}
\begin{itemize}
\item Support all kinds of frontend metaprogramming tasks
\item Especially novel tooling
\item But also def macros and macro annotations
\item More on that today in the live demo!
\end{itemize}
\end{frame}

\begin{frame}{Presentation outline}
\begin{itemize}
\item Syntactic API
\item Semantic API
\item Live demo
\item Roadmap
\end{itemize}
\end{frame}
\begin{frame}{Credits}
Big thanks to everyone who helped turning scala.meta into reality!

\begin{tabular}{p{0.4\textwidth}p{0.5\textwidth}}
\begin{itemize}
\itemsep0.5em
\item Uladzimir Abramchuk
\item Eric Beguet
\item Igor Bogomolov
\item Eugene Burmako
\item Mathieu Demarne
\item Martin Duhem
\item Adrien Ghosn
\item Vojin Jovanovic
\item Guillaume Mass\'e
\end{itemize} &
\begin{itemize}
\itemsep0.5em
\item Mikhail Mutcianko
\item Dmitry Naydanov
\item Artem Nikiforov
\item Vladimir Nikolaev
\item Martin Odersky
\item Oleksandr Olgashko
\item Alexander Podkhalyuzin
\item Jatin Puri
\item Dmitry Petrashko
\item Denys Shabalin
\end{itemize} \\
\end{tabular}
\end{frame}

\sectionframe{Part 1: Syntactic API}

\begin{frame}[fragile]{Getting started}
\begin{semiverbatim}
\$ scala

scala> import scala.meta._
import scala.meta._

scala> import scala.meta.dialects.Scala211
import scala.meta.dialects.Scala211
\end{semiverbatim}
\end{frame}

\begin{frame}{Design goals}
\begin{itemize}
\item In \texttt{scala.meta}, we keep all syntactic information about the program
\item Nothing is desugared (e.g. \texttt{for} loops or string interpolations)
\item Nothing is thrown away (e.g. comments or formatting details)
\end{itemize}
\end{frame}

\begin{frame}{Implementation vehicle}
First-class tokens
\end{frame}

\begin{frame}[fragile]
\frametitle<1-6>{Tokens}
\begin{semiverbatim}
scala> "\alert<2>{class} \alert<3>{C} \alert<4>{\{} \alert<2>{def} \alert<3>{x} \alert<4>{=} 2 \alert<4>{\}}".tokens
\only<1>{...}\visible<2->{res1: scala.meta.tokens.Tokens = Tokens(\alert<6>{BOF (0..0)},}
\visible<2->{\alert<2>{class (0..5)}, \alert<5>{(5..6)}, \alert<3>{C (6..7)}, \alert<5>{(7..8)}, \alert<4>{\{ (8..9)}, \alert<5>{(9..10)},}
\visible<2->{\alert<2>{def (10..13)}, \alert<5>{(13..14)}, \alert<3>{x (14..15)}, \alert<5>{(15..16)}, \alert<4>{= (16..17)},}
\visible<2->{\alert<5>{(17..18)}, 2 (18..19), \alert<5>{(19..20)}, \alert<4>{\} (20..21)}, \alert<6>{EOF (21..21)})}
\end{semiverbatim}
\end{frame}

\begin{frame}[fragile]{High-fidelity parsers}
\begin{semiverbatim}
scala> "class C".parse[Stat]
res2: scala.meta.Stat = class C

scala> "class C \alert<2->{\{\}}".parse[Stat]
res3: scala.meta.Stat = class C \{\}

\only<2->{scala> res2.tokens}
\only<2->{res4: scala.meta.tokens.Tokens = Tokens(BOF (0..0),}
\only<2->{class (0..5), (5..6), C (6..7), EOF(7..7))}

\only<2->{scala> res3.tokens}
\only<2->{res5: scala.meta.tokens.Tokens = Tokens(BOF (0..0),}
\only<2->{class (0..5), (5..6), C (6..7),}
\only<2->{(7..8), \alert{\{ (8..9), \} (9..10)}, EOF (10..10))}
\end{semiverbatim}
\end{frame}

\begin{frame}[fragile]{Automatic and precise range positions}
\begin{semiverbatim}
scala> "class C \{ def x = 2 \}".parse[Stat]
res6: scala.meta.Stat = class C \{ def x = 2 \}

scala> val q"class C \{ \$method \}" = res6
method: scala.meta.Stat = def x = 2

\only<2->{scala> method.tokens}
\only<2->{res6: scala.meta.tokens.Tokens = Tokens(}
\only<2->{def (10..13), (13..14), x (14..15), (15..16),}
\only<2->{= (16..17), (17..18), 2 (18..19))}
\end{semiverbatim}
\end{frame}

\begin{frame}[fragile]{Hacky quasiquotes in scala.reflect}
\begin{semiverbatim}
\$ scala -Yquasiquote-debug

scala> import scala.reflect.runtime.universe.\_
import scala.reflect.runtime.universe.\_

scala> val name = TypeName("C")
name: reflect.runtime.universe.TypeName = C

scala> q"class \alert<2->{\$name}"
...
\only<2->{code to parse:}
\only<2->{class \alert<2->{qq\$a2912896\$macro\$1}}
\only<2->{parsed:}
\only<2->{Block(List(), ClassDef(Modifiers(),}
\only<2->{\alert<2->{TypeName("qq\$a2912896\$macro\$1")}, List(), Template(...))}
\only<2->{...}
\end{semiverbatim}
\end{frame}

\begin{frame}[fragile]{Principled quasiquotes in scala.meta}
\begin{semiverbatim}
\$ scala -Dquasiquote.debug

scala> import scala.meta.\_, dialects.Scala211
import scala.meta.\_
import dialects.Scala211

scala> val name = t"C"
...
name: scala.meta.Type.Name = C

scala> q"class \alert<2->{\$name}"
...
\only<2->{Adhoc(List(BOF (0..0), class (0..5), (5..6),}
\only<2->{\alert<2->{\$name (0..5)}, EOF (0..0)))}
\only<2->{...}
\end{semiverbatim}
\end{frame}

\begin{frame}{Derived technologies}
First-class tokens enable:
\begin{itemize}
\item High-fidelity parsers
\item Automatic and precise range positions
\item Principled quasiquotes
\end{itemize}
\end{frame}

\sectionframe{Part 2: Semantic API}

\begin{frame}[fragile]{Getting started}
\begin{semiverbatim}
\$ scala

scala> import scala.meta.\_
import scala.meta.\_

scala> implicit val mirror = Mirror(...)
mirror: scala.meta.Mirror = ...
\end{semiverbatim}

\vskip25pt
\only<1>{\vskip20pt}
\only<2->{Contexts can come from:}
\begin{itemize}
\item<2-> Mirror(...): useful for standalone apps
\item<3-> Proxy(global): useful for compiler plugins
\item<4-> Anywhere else: anyone can implement a context
\end{itemize}
\end{frame}

\begin{frame}{Design goals}
\begin{itemize}
\item In \texttt{scala.meta}, we model everything just with its abstract syntax
\item Types, members, names, modifiers: all represented with trees
\item There's only one data structure, so there's only one way to do it
\end{itemize}
\end{frame}

\begin{frame}{Implementation vehicle}
First-class names
\end{frame}

\begin{frame}[fragile]{Bindings in scala.reflect}
\begin{semiverbatim}
\small
\$ scala

scala> import scala.reflect.runtime.universe.\_
import scala.reflect.runtime.universe.\_

scala> showRaw(q"class \text{\color<5>{blue}{C}} \{ def \text{\color<5>{red}{x}} = 2; def \text{\color<5>{LimeGreen}{y}} = \text{\color<5>{red}{x}} \}")
\only<1>{...}\only<2->{res1: String = \text{\color<5>{blue}{ClassDef}}(}
  \only<2->{Modifiers(), \only<1-3>{\alert<3>{TypeName}(}"C"\only<1-3>{)}, List(),}
  \only<2->{Template(}
    \only<2->{List(Select(Ident(scala), \only<1-3>{TypeName(}"AnyRef"\only<1-3>{)})),}
    \only<2->{noSelfType,}
    \only<2->{List(}
      \only<2->{DefDef(NoMods, termNames.CONSTRUCTOR, ...),}
      \only<2->{\text{\color<5>{red}{DefDef}}(NoMods, \only<1-3>{\alert<3>{TermName}(}"x"\only<1-3>{)}, ..., Literal(Constant(2))),}
      \only<2->{\text{\color<5>{LimeGreen}{DefDef}}(NoMods, \only<1-3>{\alert<3>{TermName}(}"y"\only<1-3>{)}, ..., \text{\color<5>{red}{Ident(\only<1-3>{\alert<3>{TermName}(}"x"\only<1-3>{)})}}))))}
\end{semiverbatim}
\end{frame}

\begin{frame}[fragile]{Bindings in scala.meta}
\begin{semiverbatim}
\small
\$ scala

scala> import scala.meta.\_, dialects.Scala211
import scala.meta.\_
import dialects.Scala211

scala> q"class \text{\color{blue}{C}} \{ def \text{\color{red}{x}} = 2; def \text{\color{LimeGreen}{y}} = \text{\color{red}{x}} \}".show[Structure]
res1: String = Defn.Class(
  Nil, \text{\color{blue}{Type.Name("C")}}, Nil,
  Ctor.Primary(Nil, Ctor.Name("this"), Nil),
  Template(
    Nil, Nil,
    Term.Param(Nil, Name.Anonymous(), None, None),
    Some(List(
      Defn.Def(Nil, \text{\color{red}{Term.Name("x")}}, ..., Lit.Int(2)),
      Defn.Def(Nil, \text{\color{LimeGreen}{Term.Name("y")}}, ..., \text{\color{red}{Term.Name("x")}})))))
\end{semiverbatim}
\end{frame}

\begin{frame}[fragile]{Key example}
\texttt{List[Int]}
\end{frame}

\begin{frame}[fragile]{Key example}
\begin{semiverbatim}
\small
scala> t"List[Int]".show[Structure]
res1: String =
Type.Apply(Type.Name("List"), List(Type.Name("Int")))

scala> implicit val mirror = Mirror(...)
mirror: scala.meta.Mirror = ...

\only<2>{scala> t"\text{\color{blue}{List}}[\text{\color{red}{Int}}]".show[Semantics]}
\only<2>{res3: String =}
\only<2>{Type.Apply(\text{\color{blue}{Type.Name("List")[1]}}, List(\text{\color{red}{Type.Name("Int")[2]}}))}
\only<2>{[1] \{1\}::scala.package#List}
\only<2>{[2] \{2\}::scala#Int}
\only<2>{...}
\end{semiverbatim}
\end{frame}

\begin{frame}[fragile]{Name resolution}
\begin{semiverbatim}
scala> implicit val mirror = Mirror(...)
mirror: scala.meta.Mirror = ...

scala> q"scala.collection.immutable.List".defn
res2: scala.meta.Member.Term = object List extends
SeqFactory[List] with Serializable \{ ... \}

scala> res2.name
res3: scala.meta.Term.Name = List
\end{semiverbatim}
\end{frame}

\begin{frame}[fragile]{Other semantic APIs}
\begin{semiverbatim}
scala> q"scala.collection.immutable.List".defs("apply")
res4: scala.meta.Member.Term =
override def apply[A](xs: A*): List[A] = ???

scala> q"scala.collection.immutable.List".supermembers
res5: Seq[scala.meta.Member.Term] =
List(abstract class SeqFactory...)
\end{semiverbatim}
\end{frame}

\begin{frame}{Derived technologies}
First-class names enable:
\begin{itemize}
\item Unification of trees, types and symbols
\item Referential transparency and hygiene (under development!)
\item Simpler mental model of metaprogramming
\end{itemize}
\end{frame}

\sectionframe{Part 3: Live demo}

\sectionframe{Part 4: Roadmap}

\begin{frame}{Where we've been before}
\begin{itemize}
\item With scala.meta, we started from complete scratch
\end{itemize}
\end{frame}

\begin{frame}{Lots of experimentation}
\begin{itemize}
\item Safe by construction trees
\item High-fidelity parsing
\item Automatic and precise range positions
\item Principled quasiquotes
\item Unification of trees, symbols and types
\item AST persistence
\item AST interpretation
\item Simple syntax and compilation for macros
\item IDE support for macros
\item SBT support for macros
\item ...
\end{itemize}
\end{frame}

\begin{frame}{Where we are now}
\begin{itemize}
\item Tokens provide an elegant and powerful foundation for syntactic APIs
\item Names enable a simple mental model for semantic APIs
\item People are already successfully using these new concepts!
\end{itemize}
\end{frame}

\begin{frame}{Where we will be soon}
\vskip25pt
\begin{itemize}
\item Experimentation's temporarily on hold, we're now pushing for 0.1
\item Main focus of 0.1 is making scala.meta trees publicly available
\item \only<1>{\text{\color{blue}\href{https://github.com/scalameta/scalameta/milestones/0.1}{https://github.com/scalameta/scalameta/milestones/0.1}}}\only<2>{\text{\color{black}\href{https://github.com/scalameta/scalameta/milestones/0.1}{https://github.com/scalameta/scalameta/milestones/0.1}}}
\end{itemize}
\pause
\vskip25pt
Contributor alert!\\
\text{\color{blue}\href{https://github.com/scalameta/scalameta/issues}{https://github.com/scalameta/scalameta/issues}}
\end{frame}

\sectionframe{Wrapping up}

\begin{frame}{Summary}
\begin{itemize}
\item scala.meta is a one-stop solution to frontend metaprogramming
\item Our key innovations include first-class support for tokens and names
\item We're now pushing for the 0.1 preview release
\item Join us at \text{\color{blue}\href{https://gitter.im/scalameta/scalameta}{https://gitter.im/scalameta/scalameta}}!
\end{itemize}
\end{frame}

\end{document}
