\documentclass[svgnames,dvipsnames,hyperref={bookmarks=false},usepdftitle=false]{beamer}
\usetheme{scala2014}

\title{MorphScala: Safe Class Morphing with Macros}
\author{Aggelos Biboudis (\href{https://twitter.com/biboudis}{@biboudis})\\ Eugene Burmako (\href{https://twitter.com/xeno_by}{@xeno{\textunderscore}by})}
\institute{University of Athens\\ \'Ecole Polytechnique F\'ed\'erale de Lausanne}
\date{29 July 2014}
\hypersetup{pdfauthor={Aggelos Biboudis, Eugene Burmako},pdftitle={Easy Metaprogramming For Everyone!}}

\begin{document}

\titleframe

\begin{frame}{Outline}
\begin{itemize}
\item Motivation for class morphing in Scala
\item The idea presented in the paper (the implementation is not there yet!)
\item Future implementation steps
\end{itemize}
\end{frame}

\sectionframe{Motivation}

\begin{frame}{Scala macros}
\begin{itemize}
\item Experimental language feature available for almost 2 years
\item Multiple flavors: def macros, type macros, macro annotations
\item Most widely used are def macros: methods whose applications expand on sub-method level
\end{itemize}
\end{frame}

\begin{frame}[fragile]{Blackbox macros}
\begin{semiverbatim}
trait Query[T] \{
  \alert{def filter(p: T => Boolean): Query[T] = macro ...}
\}

val users: Query[User] = ...
users\alert{.filter(}_.name == "John"\alert{)}


                          \arrowdown

Query(Filter(users, Equals(Ref("name"), Literal("John"))))

\end{semiverbatim}

\begin{itemize}
\item Look like normal methods with honest type signatures
\item Can be treated as black boxes by machines and humans alike
\end{itemize}
\end{frame}

\begin{frame}[fragile]{Whitebox macros}
\begin{semiverbatim}
\alert{def h2db(connString: String): Any = macro ...}
val db = \alert{h2db(}"jdbc:h2:coffees.h2.db"\alert{)}

                          \arrowdown

val db = \{
  trait Db \{
    case class Coffee(...)
    val Coffees: Table[Coffee] = ...
  \}
  new Db \{\}
\}

\end{semiverbatim}

\begin{itemize}
\item Can refine their return types
\item Thanks to first-class modules this can generate public definitions
\end{itemize}
\end{frame}

\begin{frame}{The whitebox conundrum}
\begin{itemize}
\item Practice shows that type signatures are very important for macros
\item But practice also demands intra-method code generation
\item Blackbox can't do the latter, whitebox isn't very good at the former
\item Stalemate?
\end{itemize}
\end{frame}

\begin{frame}{A new hope}
\begin{itemize}
\item At last year's OOPSLA, Aggelos introduced me to MorphJ
\item MorphJ is an extension to Java that enables class morphing, a form of intra-method template metaprogramming (TMP)
\item TMP is an interesting declarative approach to metaprogramming, but most importantly MorphJ's TMP allows modular typechecking
\item So we decided to give it a try
\end{itemize}
\end{frame}

\sectionframe{The idea}

\begin{frame}[fragile]{\only<1>{Metaclasses}\only<2>{Compile-time reflection}\only<3>{(1) Uniqueness of declarations}\only<4>{(2) Validity of references}}
\begin{semiverbatim}
\alert<1>{@morph}
class Logged[X] extends X \{
  \alert<2>{for (q"def \$m(..\$params): \$r" <- members[X]) \{}
    val args = params.map(p => q"\$\{p.name\}")
    q"""\alert<2,3>{override def \$m(..\$params): \$r} = \{
          val result = \alert<4>{super.\$m}(..\$args)
          \alert<4>{println}(result)
          result
        \}"""
  \}
\}

\end{semiverbatim}

\begin{itemize}
\only<1>{\item \texttt{@morph} designates a metaclass for \texttt{Logged[X]} classes}
\only<1>{\item \texttt{Logged[X]} classes are instantiated and used in the normal fashion}
\only<2>{\item MorphScala provides a quasiquote-based DSL to iterate members}
\only<2>{\item Members can be destructured and give rise to new members}
\only<3-4>{\item A very appealing property of morphing is modular type checking}
\only<3>{\item First, we want to guarantee that generated members don't overlap}
\only<4>{\item Second, we want to guarantee that all references are valid}
\end{itemize}
\end{frame}

\sectionframe{Conclusion}

\begin{frame}{Status}
\begin{itemize}
\item We haven't implemented the outlined idea yet
\item There are still some open questions
\item However most of the implementation tools are already in place
\end{itemize}
\end{frame}

\begin{frame}[fragile]{\only<1>{(1) Restrictions for the compile-time reflection API}\only<2>{(2) Implementation vehicles}}
\begin{semiverbatim}
\alert<2>{@morph}
class Logged[X] extends X \{
  for (q"def \$m(..\$params): \$r" <- members[X]) \{
    \alert<1>{val args = params.map(p => q"\$\{p.name\}")}
    q"""override def \$m(..\$params): \$r = \{
          val result = super.\$m(..\$args)
          println(result)
          result
        \}"""
  \}
\}

\end{semiverbatim}

\begin{itemize}
\only<1>{\item Modular type checking is very important}
\only<1>{\item How much flexibility do we have to sacrifice to achieve it?}
\only<1>{\item What constructs can we safely intrinsify?}
\only<2>{\item Can be implemented by a combo of type macros and macro annots}
\only<2>{\item Are we happy about this elaborate scheme?}
\only<2>{\item How do we revive type macros?}
\end{itemize}
\end{frame}

\begin{frame}{Summary}
\begin{itemize}
\item MorphScala = template metaprogramming facility inspired by MorphJ
\item Emphasis on strong type checking guarantees (modular type checking)
\item Implementation is in the works
\end{itemize}
\end{frame}

\end{document}