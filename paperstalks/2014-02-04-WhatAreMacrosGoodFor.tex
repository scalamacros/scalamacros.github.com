\documentclass[svgnames,hyperref={bookmarks=false}]{beamer}
\useoutertheme{infolines}
\setbeamertemplate{headline}{} % removes the headline that infolines inserts
% \setbeamertemplate{footline}{
%   \hfill%
%   \usebeamercolor[fg]{page number in head/foot}%
%   \usebeamerfont{page number in head/foot}%
%   \insertpagenumber\,/\,\insertpresentationendpage\kern1em\vskip2pt%
% }
\setbeamertemplate{footline}{
  \hfill%
  \usebeamercolor[fg]{page number in head/foot}%
  \usebeamerfont{page number in head/foot}%
  \footnotesize\insertpagenumber\kern1em\vskip2pt%
}
\setbeamertemplate{navigation symbols}{}
\setbeamercolor{alerted text}{fg=blue}
\setbeamerfont{alerted text}{series=\bfseries,family=\ttfamily}
\setbeamertemplate{section in toc}{\text{\color{black}{\inserttocsection}}}
\usepackage[parfill]{parskip}
\usepackage[linewidth=0.5pt]{mdframed}
\newmdenv[innerleftmargin=1mm, innerrightmargin=1mm, innertopmargin=-1mm, innerbottommargin=2mm, leftmargin=-1mm, rightmargin=-1mm]{lstlistinglike}
\usepackage{tikz}
\usepackage{graphicx}
\DeclareGraphicsExtensions{.pdf,.png,.jpg}
\usepackage{textcomp}
\usepackage{pifont}
\usetikzlibrary{shapes.arrows}

\tikzset{
    myarrow/.style={
        draw,
        fill=orange,
        single arrow,
        minimum height=3.5ex,
        single arrow head extend=1ex
    }
}
\newcommand{\arrowup}{%
\tikz [baseline=-0.5ex]{\node [myarrow,rotate=90] {};}
}
\newcommand{\arrowdown}{%
\tikz [baseline=-1ex]{\node [myarrow,rotate=-90] {};}
}

\hypersetup{pdfauthor={Eugene Burmako},pdfsubject={What Are Macros Good For?},pdftitle={What Are Macros Good For?}}
\title{What Are Macros Good For?}

\begin{document}

\title{What Are Macros Good For?}
\author{Eugene Burmako}
\institute{\'Ecole Polytechnique F\'ed\'erale de Lausanne \\
           \texttt{http://scalamacros.org/}}
\date{04 February 2014}
{
\setbeamertemplate{footline}{}
\begin{frame}
  \titlepage
\end{frame}
}

\begin{frame}[fragile]
\frametitle{What are macros?}

\begin{itemize}
\item An experimental feature of 2.10 and 2.11
\item You write functions against the reflection API
\item Compiler invokes them during compilation
\end{itemize}
\end{frame}

\begin{frame}[fragile]
\frametitle{Macro flavors}

\begin{itemize}
\item Many ways to hook into the compiler \text{\textrightarrow} many macro flavors
\item Type macros, annotation macros, untyped macros, etc
\item However in 2.10 and 2.11 there are only def macros
\end{itemize}
\end{frame}

\begin{frame}[fragile]
\frametitle{Def macros}

\begin{semiverbatim}
\alert{log(}"does not compute"\alert{)}

                          \arrowdown

if (Logger.enabled)
  Logger.log("does not compute")

\end{semiverbatim}

\begin{itemize}
\item Def macros replace well-typed terms with other well-typed terms
\item Generated code can contain arbitrary Scala constructs
\item Code generation can involve arbitrary computations
\end{itemize}
\end{frame}

\begin{frame}[t, fragile]
\frametitle{Def macros}

\begin{semiverbatim}
\alert{def log(msg: String): Unit}\only<1>{ \alert{= ...}





}\only<2->{ \alert{= macro impl}}

\only<2->{def impl(c: Context)(msg: c.Expr[String]): c.Expr[Unit]}\only<2>{ = ...





}\only<3>{ = \{
  import c.universe._




\}}\only<4->{ = \{
  import c.universe._
  q"""
    if (Logger.enabled)
      Logger.log(\$msg)
  """
\}}



\end{semiverbatim}

\begin{itemize}
\item Macro signatures look like signatures of normal methods
\only<2->{\item Macro bodies are just stubs, referring macro impls defined outside}
\only<3->{\item Implementations use reflection API to analyze and generate code}
\end{itemize}
\end{frame}

\begin{frame}[fragile]
\frametitle{Quasiquotes}

\begin{semiverbatim}
q"""
  if (Logger.enabled)
    Logger.log(\$msg)
"""

\end{semiverbatim}

\begin{itemize}
\item \texttt{q"..."} string interpolators that build code are called quasiquotes
\item They are very convenient to create and pattern match code snippets
\item In 2.10 quasiquotes are available via \text{\color{blue}\href{http://docs.scala-lang.org/overviews/macros/paradise.html}{the macro paradise plugin}}
\item In 2.11 quasiquotes are available in the standard Scala distribution
\end{itemize}

\end{frame}

\begin{frame}[fragile]
\frametitle{Summary}

\begin{semiverbatim}
\alert{log(}"does not compute"\alert{)}

                          \arrowdown

if (Logger.enabled)
  Logger.log("does not compute")

\end{semiverbatim}
\begin{itemize}
\item Local expansion of method calls
\item Well-formed and well-typed arguments
\item Now what is this good for?
\end{itemize}
\end{frame}

\begin{frame}[fragile]
\frametitle{}

\vskip40pt
\begin{center}
\text{\color{blue}{\Large{Code generation}}}
\end{center}
\end{frame}

\begin{frame}[fragile]
\frametitle{Code generation}

\begin{itemize}
\item Create terms and types on-the-fly
\item More convenient and robust than textual codegen
\end{itemize}
\end{frame}

\begin{frame}[fragile]
\frametitle{Example \#1 - Term generation}

\begin{semiverbatim}
def createArray[T: ClassTag](size: Int, el: T) = \{
  val a = new Array[T](size)
  for (i <- 0 until size) a(i) = el
  a
\}

\end{semiverbatim}

\begin{itemize}
\item We want to write beautiful generic code, and Scala makes that easy
\item Unfortunately, abstractions oftentimes bring overhead
\item E.g. in this case erasure will cause boxing leading to a slowdown
\end{itemize}
\end{frame}

\begin{frame}[fragile]
\frametitle{Example \#1 - Term generation}

\begin{semiverbatim}
def createArray[@specialized T: ClassTag](...) = \{
  val a = new Array[T](size)
  for (i <- 0 until size) a(i) = el
  a
\}

\end{semiverbatim}

\begin{itemize}
\item Methods can be \texttt{@specialized}, but it's viral and heavyweight
\item Viral = the entire call chain needs to be specialized
\item Heavyweight = specialization leads to duplication of bytecode
\end{itemize}
\end{frame}

\begin{frame}[fragile]
\frametitle{Example \#1 - Term generation}

\begin{semiverbatim}
def createArray[T: ClassTag](size: Int, el: T) = \{
  val a = new Array[T](size)
  def specBody[@specialized T](el: T) \{
    for (i <- 0 until size) a(i) = el
  \}
  classTag[T] match \{
    case ClassTag.Int => specBody(el.asInstanceOf[Int])
    ...
  \}
  a
\}
\end{semiverbatim}

\begin{itemize}
\item We want to specialize just as much as we need
\item As described in the recent \text{\color{blue}\href{http://lampwww.epfl.ch/\~hmiller/scala2013/resources/pdfs/paper10.pdf}{Bridging Islands of Specialized Code}} paper
\item But that's tiresome to do by hand, and this is where macros shine
\end{itemize}
\end{frame}

\begin{frame}[fragile]
\frametitle{Example \#1 - Term generation}

\begin{semiverbatim}
\alert{def specialized[T: ClassTag](code: => Any) = macro ...}

def createArray[T: ClassTag](size: Int, el: T) = \{
  val a = new Array[T](size)
  \alert{specialized[T] \{}
    for (i <- 0 until size) a(i) = el
  \alert{\}}
  a
\}

\end{semiverbatim}

\begin{itemize}
\item \texttt{specialized} macro gets pretty code and transforms it into fast code
\item This is a typical scenario of using macros for performance
\item Also see the talk on \text{\color{blue}\href{http://skillsmatter.com/podcast/scala/macro-based-scala-parallel-collections}{Macro-Based Scala Parallel Collections}}
\end{itemize}
\end{frame}

\begin{frame}[fragile]
\frametitle{Example \#2 - Type generation}

\begin{semiverbatim}
println(Db.Coffees.all)
Db.Coffees.insert("Brazilian", 99, 0)

\end{semiverbatim}

\begin{itemize}
\item In F\# one can generate wrappers over datasources
\item These wrappers can then be used in a strongly-typed manner
\item Can this be implemented with def macros?
\end{itemize}
\end{frame}

\begin{frame}[fragile, t]
\frametitle{Example \#2 - Type generation}

\begin{semiverbatim}
\alert{def h2db(connString: String): Any = macro ...}
val db = \alert{h2db(}"jdbc:h2:coffees.h2.db"\alert{)}

                          \arrowdown

val db = \{
  trait Db \{
    case class Coffee(...)
    val Coffees: Table[Coffee] = ...
  \}
  new Db \{\}
\}

\end{semiverbatim}

\begin{itemize}
\item Def macros expand locally, therefore we get a bunch of local classes
\item Locals are invisible from the outside, so it's a game over? Nope!
\end{itemize}
\end{frame}

\begin{frame}[fragile, t]
\frametitle{Example \#2 - Type generation}

\begin{semiverbatim}
scala> val db = \alert{h2db(}"jdbc:h2:coffees.h2.db"\alert{)}
db: AnyRef \{
  type Coffee \{ val name: String; val price: Int; ... \}
  val Coffees: Table[this.Coffee]
\} = \$anon\$1...

scala> db.Coffees.all
res1: List[Db\$1.this.Coffee] = List(Coffee(Brazilian,99,0))

\end{semiverbatim}

\begin{itemize}
\item Scala can figure out and expose local signatures to the outer world
\item Used by Specs2 to automatically create matchers for custom classes
\end{itemize}
\end{frame}

\begin{frame}[fragile]
\frametitle{Example \#2 - Type generation}

\begin{semiverbatim}
scala> val db = \alert{h2db(}"jdbc:h2:coffees.h2.db"\alert{)}
db: \{ type Coffee \{ ... \}; val Coffees: List[this.Coffee]; \}

\end{semiverbatim}

\begin{itemize}
\item This is a fun technique stretching the boundaries of macrology
\item There are \text{\color{blue}\href{http://docs.scala-lang.org/overviews/macros/typeproviders.html}{some caveats}}, so it should be used with caution
\item Alternatively you could use macro annotations available in 2.10 and 2.11 via the macro paradise plugin
\end{itemize}
\end{frame}

\begin{frame}[fragile]
\frametitle{Example \#3 - Materialization}

\begin{semiverbatim}
trait Reads[T] \{
  def reads(json: JsValue): JsResult[T]
\}

object Json \{
  def fromJson[T](json: JsValue)
    (implicit fjs: Reads[T]): JsResult[T]
\}

\end{semiverbatim}

\begin{itemize}
\item Type classes are an idiomatic way of writing extensible code in Scala
\item This is an example of typeclass-based design in Play
\end{itemize}
\end{frame}

\begin{frame}[fragile]
\frametitle{Example \#3 - Materialization}

\begin{semiverbatim}
def fromJson[T](json: JsValue)
  (implicit fjs: Reads[T]): JsResult[T]

implicit val IntReads = new Reads[Int] \{
  def reads(json: JsValue): JsResult[T] = ...
\}

fromJson[Int](json) // you write
fromJson[Int](json)(IntReads) // you get

\end{semiverbatim}

\begin{itemize}
\item With type classes we externalize the moving parts
\item Instances of type classes are provided once
\item And then \texttt{scalac} fills them in automatically
\end{itemize}
\end{frame}

\begin{frame}[fragile]
\frametitle{Example \#3 - Before macros}

\begin{semiverbatim}
case class Person(name: String, age: Int)

implicit val personReads = (
  (__ \\ 'name).reads[String] and
  (__ \\ 'age).reads[Int]
)(Person)

\end{semiverbatim}

\begin{itemize}
\item Everything is done manually, hence boilerplate
\item There are alternatives, e.g. \text{\color{blue}\href{https://speakerdeck.com/larsrh/generating-type-class-instances-automatically}{one presented at the Scala'13 workshop}}
\item But each of them has its downsides
\end{itemize}
\end{frame}

\begin{frame}[fragile]
\frametitle{Example \#3 - Vanilla macros (2.10.0)}

\begin{semiverbatim}
implicit val personReads = Json\alert{.reads[}Person\alert{]}

\end{semiverbatim}

\begin{itemize}
\item Boilerplate can be generated by a macro
\item The code ends up being the same as if it were written manually
\item Therefore performance remains excellent
\end{itemize}
\end{frame}

\begin{frame}[fragile]
\frametitle{Example \#3 - Implicit macros (2.10.2+)}

\begin{semiverbatim}
// no code necessary

\end{semiverbatim}

\begin{itemize}
\item Implicit values can be transparently generated by implicit macros
\item Used with success in pickling and shapeless
\end{itemize}
\end{frame}

\begin{frame}[fragile]
\frametitle{Example \#3 - Implicit macros (2.10.2+)}

\begin{semiverbatim}
trait Reads[T] \{ def reads(json: JsValue): JsResult[T] \}

object Reads \{
  \alert{implicit def materializeReads[T]: Reads[T] = macro ...}
\}

\end{semiverbatim}

\begin{itemize}
\item When \texttt{scalac} looks for implicits, it traverses the implicit scope
\item Implicit scope transcends lexical scope
\item Among others it includes members of the target’s companion
\end{itemize}
\end{frame}

\begin{frame}[fragile]
\frametitle{Example \#3 - Implicit macros (2.10.2+)}

\begin{semiverbatim}
fromJson[Person](json)

                          \arrowdown

fromJson[Person](json)(\alert{materializeReads[}Person\alert{]})

                          \arrowdown

fromJson[Person](json)(new Reads[Person]\{ ... \})

\end{semiverbatim}

\begin{itemize}
\item Every time a \texttt{Reads[T]} isn't found, the compiler will call our macro
\item Details on how this works can be found in \text{\color{blue}\href{http://docs.scala-lang.org/overviews/macros/implicits.html}{our documentation}}
\end{itemize}
\end{frame}

\begin{frame}[fragile]
\frametitle{}

\vskip40pt
\begin{center}
\text{\color{blue}{\Large{Static checks}}}
\end{center}
\end{frame}

\begin{frame}[fragile]
\frametitle{Static checks}

\begin{itemize}
\item Check your program during compilation
\item Report errors and warnings as you go
\end{itemize}
\end{frame}

\begin{frame}[fragile, t]
\frametitle{Example \#4 - Advanced type signatures}

\begin{semiverbatim}
trait Request
case class Command(msg: String) extends Request

trait Reply
case object CommandSuccess extends Reply
case class CommandFailure(msg: String) extends Reply

\only<1>{val actor = someActor}\only<2>{type Spec = (Request, Reply) :+: TNil}
\only<1>{actor ! Command}\only<2>{val actor = new ChannelRef[Spec](someActor)}
\alert{\only<2>{actor <-!- Command // doesn't compile}}

\end{semiverbatim}

\begin{itemize}
\only<1>{\item Akka actors are dynamically typed, i.e. the \texttt{!} method takes \texttt{Any}}
\only<1>{\item This loosens type guarantees provided by Scala}
\only<1>{\item E.g. here we have a sneaky type error that leads to a runtime crash}
\only<2>{\item We can implement type specification for actors even in standard Scala}
\only<2>{\item But this became practical only when we got macros}
\only<2>{\item Akka typed channels are specifically designed to make use of macros}
\end{itemize}
\end{frame}

\begin{frame}[fragile]
\frametitle{Example \#4 - Advanced type signatures}

\begin{semiverbatim}
type Spec = (Request, Reply) :+: TNil
val actor = new ChannelRef[Spec](someActor)
\alert{actor <-!- Command // doesn't compile}

\end{semiverbatim}

\begin{itemize}
\only<1>{\item The \texttt{<-!-} macro takes the type of its target and extracts the spec}
\only<1>{\item Then it takes the argument type and validates it against the spec}
\only<1>{\item If necessary, the macro produces precise and clear compilation errors}
\only<2>{\item This all can be done with implicits and type-level computations}
\only<2>{\item But that's non-trivial both for the library authors and for the users}
\only<2>{\item Macros aren't ideal either, and we plan to further research this}
\end{itemize}
\end{frame}

\begin{frame}[fragile, t]
\frametitle{Example \#5 - Advanced static checks}

\begin{semiverbatim}
def future[T](body: => T) = ...

def receive = \{
  case Request(data) =>
    future \{
      val result = transform(data)
      sender ! Response(result)
    \}
\}

\end{semiverbatim}

\begin{itemize}
\item Capturing \texttt{sender} in the above closure is dangerous
\item That's because \texttt{sender} is not a value, but a stateful method
\item To validate captures we can use macros: \text{\color{blue}\href{http://docs.scala-lang.org/sips/pending/spores.html}{SIP-21 \text{\textendash} Spores}}
\end{itemize}
\end{frame}

\begin{frame}[fragile, t]
\frametitle{Example \#5 - Advanced static checks}

\begin{semiverbatim}
def future[T](body: \alert{Spore[T]}) = ...

\alert{def spore[T](body: => T): Spore[T] = macro ...}

def receive = \{
  case Request(data) =>
    future(\alert{spore \{}
      val result = transform(data)
      sender ! Response(result) \alert{// doesn't compile}
    \alert{\}})
\}

\end{semiverbatim}

\begin{itemize}
\item The \texttt{spore} macro takes its body and figures out all free variables
\item If any of the free variables are deemed dangerous, an error is reported
\end{itemize}
\end{frame}

\begin{frame}[fragile, t]
\frametitle{Example \#5 - Advanced static checks}

\begin{semiverbatim}
def future[T](body: Spore[T]) = ...

\alert{implicit def anyToSpore[T](body: => T): Spore[T] = macro ...}

def receive = \{
  case Request(data) =>
    future \alert{\{}
      val result = transform(data)
      sender ! Response(result) \alert{// doesn't compile}
    \alert{\}}
\}

\end{semiverbatim}

\begin{itemize}
\item The conversion to \texttt{Spore} can be made implicit
\item That will verify closures without bothering the user
\end{itemize}
\end{frame}

\begin{frame}[fragile]
\frametitle{}

\vskip40pt
\begin{center}
\text{\color{blue}{\Large{Domain-specific languages}}}
\end{center}
\end{frame}

\begin{frame}[fragile]
\frametitle{Domain-specific languages}

\begin{itemize}
\item Take a program written in an internal or external DSL
\item Work with it as with a domain-specific data structure
\end{itemize}
\end{frame}

\begin{frame}[fragile, t]
\frametitle{Example \#6 - Language virtualization}

\begin{semiverbatim}
val usersMatching = query[String, (Int, String)](
  "select id, name from users where name = ?")
usersMatching("John")

\only<2->{case class User(id: Column[Int], name: Column[String])}
\only<2->{users.filter(_.name === "John")}

\only<3->{case class User(id: Int, name: String)}
\only<3->{users\alert{.filter(}_.name == "John"\alert{)}}

\end{semiverbatim}

\begin{itemize}
\item Database queries can be written in SQL
\only<2->{\item They can also be written in a DSL, at times slightly awkward}
\only<3->{\item Or they can be written in Scala and virtualized by a macro}
\end{itemize}
\end{frame}

\begin{frame}[fragile]
\frametitle{Example \#6 - Language virtualization}

\begin{semiverbatim}
trait Query[T] \{
  \alert{def filter(p: T => Boolean): Query[T] = macro ...}
\}

val users: Query[User] = ...
users\alert{.filter(}_.name == "John"\alert{)}


                          \arrowdown

Query(Filter(users, Equals(Ref("name"), Literal("John"))))

\end{semiverbatim}

\begin{itemize}
\item The \texttt{filter} macro takes an AST corresponding to the predicate
\item This AST is then analyzed and transformed into a query fragment
\item Now we have a deeply embedded DSL, just like in LINQ and Slick
\end{itemize}
\end{frame}

\begin{frame}[fragile, t]
\frametitle{Example \#7 - Internal DSLs}

\begin{semiverbatim}
val futureDOY: Future[Response] =
  WS.url("http://api.day-of-year/today").get

val futureDaysLeft: Future[Response] =
  WS.url("http://api.days-left/today").get

futureDOY.flatMap \{ doyResponse =>
  val dayOfYear = doyResponse.body
  futureDaysLeft.map \{ daysLeftResponse =>
    val daysLeft = daysLeftResponse.body
    Ok(s"\$dayOfYear: \$daysLeft days left!")
  \}
\}
\end{semiverbatim}

\begin{itemize}
\item Turning a synchronous program into an async one isn't easy
\item One has to manually manage callbacks, introduce temps, etc
\end{itemize}
\end{frame}

\begin{frame}[fragile, t]
\frametitle{Example \#7 - Internal DSLs}

\begin{semiverbatim}
\alert{def async[T](body: => T): Future[T] = macro ...}
\alert{def await[T](future: Future[T]): T = macro ...}

\alert{async \{}
  val dayOfYear = \alert{await(}futureDOY\alert{)}.body
  val daysLeft = \alert{await(}futureDaysLeft\alert{)}.body
  Ok(s"\$dayOfYear: \$daysLeft days left!")
\alert{\}}



\end{semiverbatim}

\begin{itemize}
\item Turning a synchronous program into an async one isn't easy
\item Macros can do the transformation automatically: \text{\color{blue}\href{http://docs.scala-lang.org/sips/pending/async.html}{SIP-22 \text{\textendash} Async}}
\item Similar to C\#'s \texttt{async/await} and parts of Clojure's \texttt{core/async}
\end{itemize}
\end{frame}

\begin{frame}[fragile]
\frametitle{Example \#7 - Internal DSLs}

\begin{semiverbatim}
\alert{def async[T](body: => T): Future[T] = macro ...}
\alert{def await[T](future: Future[T]): T = macro ...}

\end{semiverbatim}

\begin{itemize}
\item At the heart of macro-based DSLs is the ability to analyze code
\item The \texttt{async} macro sees detailed inner structure of code representing its argument and can transform that structure to its liking
\item Also see today's talk \text{\color{blue}\href{http://skillsmatter.com/podcast/scala/jscala-write-your-javascript-in-scala}{JScala - Write Your JavaScript In Scala}}
\end{itemize}
\end{frame}

\begin{frame}[fragile]
\frametitle{Example \#8 - External DSLs}

\begin{semiverbatim}
scala> val x = "42"
x: String = 42

scala> "%d".format(x)
j.u.IllegalFormatConversionException: d != java.lang.String
  at java.util.Formatter\$FormatSpecifier.failConversion...
\only<2>{
scala> \alert{f"}\$x\%d\alert{"}
<console>:31: error: type mismatch;
 found   : String
 required: Int
}
\end{semiverbatim}

\begin{itemize}
\item Strings are typically perceived to be unsafe
\only<2>{\item Though with macros they don't have to be}
\end{itemize}
\end{frame}

\begin{frame}[fragile]
\frametitle{Example \#8 - External DSLs}

\begin{semiverbatim}
implicit class Formatter(c: StringContext) \{
  \alert{def f(args: Any*): String = macro ???}
\}

val x = "42"
\alert{f"}\$x\%d\alert{"} // rewritten into: StringContext("", "%d").\alert{f(}x\alert{)}

\end{semiverbatim}

\begin{itemize}
\item String interpolation desugars custom string literals into method calls
\item These methods can be macros that validate strings at compile-time
\end{itemize}
\end{frame}

\begin{frame}[fragile]
\frametitle{Example \#8 - External DSLs}

\begin{semiverbatim}
val x = "42"
\alert{f"}\$x\%d\alert{"} // rewritten into: StringContext("", "%d").\alert{f(}x\alert{)}

                          \arrowdown

\{
  val arg\$1: Int = x \alert{// doesn't compile}
  "%d".format(arg\$1)
\}

\end{semiverbatim}

\begin{itemize}
\item Here the \texttt{f} macro just inserts type ascriptions in strategic places
\item But this approach can be used to embed much more complex DSLs
\item This means static validation, typechecking and maybe even interop
\end{itemize}
\end{frame}

\begin{frame}[fragile]
\frametitle{}

\vskip40pt
\begin{center}
\text{\color{blue}{\Large{Summary}}}
\end{center}
\end{frame}

\begin{frame}[fragile]
\frametitle{What are macros good for?}

\begin{itemize}
\item Code generation
\item Static checks
\item Domain-specific languages
\end{itemize}
\end{frame}

\end{document}
