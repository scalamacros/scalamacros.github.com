\documentclass[svgnames,hyperref={bookmarks=false}]{beamer}
\useoutertheme{infolines}
\setbeamertemplate{headline}{} % removes the headline that infolines inserts
% \setbeamertemplate{footline}{
%   \hfill%
%   \usebeamercolor[fg]{page number in head/foot}%
%   \usebeamerfont{page number in head/foot}%
%   \insertpagenumber\,/\,\insertpresentationendpage\kern1em\vskip2pt%
% }
\setbeamertemplate{footline}{
  \hfill%
  \usebeamercolor[fg]{page number in head/foot}%
  \usebeamerfont{page number in head/foot}%
  \footnotesize\insertpagenumber\kern1em\vskip2pt%
}
\setbeamertemplate{navigation symbols}{}
\setbeamercolor{alerted text}{fg=blue}
\setbeamerfont{alerted text}{series=\bfseries,family=\ttfamily}
\setbeamertemplate{section in toc}{\text{\color{black}{\inserttocsection}}}
\usepackage[parfill]{parskip}
\usepackage[linewidth=0.5pt]{mdframed}
\newmdenv[innerleftmargin=1mm, innerrightmargin=1mm, innertopmargin=-1mm, innerbottommargin=2mm, leftmargin=-1mm, rightmargin=-1mm]{lstlistinglike}
\usepackage{tikz}
\usepackage{graphicx}
\DeclareGraphicsExtensions{.pdf,.png,.jpg}
\usepackage{textcomp}
\usepackage{pifont}
\usetikzlibrary{shapes.arrows}

\tikzset{
    myarrow/.style={
        draw,
        fill=orange,
        single arrow,
        minimum height=3.5ex,
        single arrow head extend=1ex
    }
}
\newcommand{\arrowup}{%
\tikz [baseline=-0.5ex]{\node [myarrow,rotate=90] {};}
}
\newcommand{\arrowdown}{%
\tikz [baseline=-1ex]{\node [myarrow,rotate=-90] {};}
}

\hypersetup{pdfauthor={Eugene Burmako},pdfsubject={Half a Year in Macro Paradise},pdftitle={Half a Year in Macro Paradise}}
\title{Half a Year In Macro Paradise}

\begin{document}

\title{Half a Year in Macro Paradise}
\author{Eugene Burmako}
\institute{\'Ecole Polytechnique F\'ed\'erale de Lausanne \\
           \texttt{http://scalamacros.org/}}
\date{12 June 2013}
{
\setbeamertemplate{footline}{}
\begin{frame}
  \titlepage
\end{frame}
}

\begin{frame}[fragile]
\frametitle{What this talk covers}

\begin{itemize}
\item New developments in macros after 2.10.0
\item Reflection on our experience with macros
\item The future of macros in Scala 2.10+
\end{itemize}
\end{frame}

\begin{frame}[fragile]
\frametitle{What this talk doesn't cover}

\begin{itemize}
\item New developments powered by macros
\begin{itemize}
\item Pickles and spores (Heather's talk \text{\color{blue}\href{http://www.scaladays.org/\#/june-12/room2/13:30-14:15/On-Pickles-and-Spores\%3A-Improving-Scala\%27s-Support-for-Distributed-Programming}{today at 13:30}})
\item scala-async (Philipp's and Jason's talk \text{\color{blue}\href{http://www.scaladays.org/\#/june-12/room1/14:30-15:15/Scala-Async\%3A-A-New-Way-to-Simplify-Asynchronous-Code-(Make-the-Compiler-Do-It!)}{today at 14:30}})
\item shapeless (Miles' talk \text{\color{blue}\href{http://www.scaladays.org/\#/june-12/room3/14:30-15:15/Expanding-eta-expansion\%3A-shapeless-polymorphic-function-values-meet-macros}{today at 14:30}})
\item scala-workflow (Evgeny's project \text{\color{blue}\href{https://github.com/aztek/scala-workflow/}{at GitHub}})
\item Akka typed channels (the video of Roland presenting \text{\color{blue}\href{http://marakana.com/s/post/1407/akka_typed_channels_implementing_type_calculations_as_macros_scala_video}{at NEScala}})
\item Yin-Yang (Vojin's paper \text{\color{blue}\href{http://infoscience.epfl.ch/record/185832}{at infoscience.epfl.ch}})
\item Specialization 2.0 (Nicolas' and Vlad's project \text{\color{blue}\href{https://github.com/nicolasstucki/specialized}{at GitHub}})
\item Type-safe JSON (Greg's talk \text{\color{blue}\href{http://2013.geecon.org/speakers/grzegorz-kossakowski}{at Geecon}})
\item Improvements for the cake pattern (John Sullivan's talk \text{\color{blue}\href{http://www.scaladays.org/\#/june-12/room1/11:15-12:00/Taming-the-Cake-Pattern-with-Type-Macros}{today at 11:15}})
\item Parallel collections 2.0 (coming \text{\color{blue}\href{http://axel22.github.io/home/professional/\#projects}{coming this summer}})
\end{itemize}
\end{itemize}
\end{frame}

\begin{frame}[fragile]
\frametitle{What this talk doesn't cover}

\begin{itemize}
\item New developments powered by macros (see the previous slide)
\item Best practices (\text{\color{blue}\href{http://scalapeno.underscore.co.il/\#!macro-writers-guide-210x-and-beyond/c134f}{my upcoming talk at Scalape\~{n}o}})
\item Design details (\text{\color{blue}\href{https://thestrangeloop.com/sessions/evolution-of-scala-macros}{my upcoming talk at Strange Loop}})
\end{itemize}
\end{frame}

\begin{frame}[fragile]
\frametitle{}

\vskip40pt
\begin{center}
\text{\color{blue}{\Large{Macros in Scala 2.10}}}
\end{center}
\end{frame}

\begin{frame}[fragile]
\frametitle{Macros}

\begin{itemize}
\item New experimental feature in Scala 2.10.0
\item Macros are functions written in Scala against reflection API
\item They are invoked by the compiler during compilation
\item A lot of cool things can be done with a compiler API, so there are multiple macro flavors
\end{itemize}
\end{frame}

\begin{frame}[fragile]
\frametitle{Def macros}

\begin{itemize}
\item The only macro flavor in Scala 2.10.0
\item Calls to def macros expand into programmatically generated code
\item \text{\color{blue}\href{http://docs.scala-lang.org/overviews/macros/overview.html}{http://docs.scala-lang.org/overviews/macros/overview.html}}
\end{itemize}
\end{frame}

\begin{frame}[fragile]
\frametitle{Example}

\begin{semiverbatim}
log(Error, "does not compute")

                          \arrowdown

if (Config.loggingEnabled)
  Config.logger.log(Error, "does not compute")

\end{semiverbatim}

\begin{itemize}
\item We will now write a macro that automates logging
\item Without macros this is impossible to achieve at zero performance cost
\end{itemize}
\end{frame}

\begin{frame}[t, fragile]
\frametitle{Example}

\begin{semiverbatim}
def log(severity: Severity, msg: String): Unit\only<1>{ = ...







}\only<2->{ = macro impl}

\only<2->{def impl(c: Context)
    (severity: c.Expr[Severity],
     msg: c.Expr[String]): c.Expr[Unit]}\only<2>{ = ...





}\only<3->{ = \{
  import c.universe._
  reify \{
    if (Config.loggingEnabled)
      Config.logger.log(severity.splice, msg.splice)
  \}
\}}

\end{semiverbatim}

\begin{itemize}
\item Macro signatures look like signatures of normal methods
\only<2->{\item Macro bodies are just stubs, implementations are defined outside}
\only<3->{\item Implementations use reflection API to analyze and generate code}
\end{itemize}
\end{frame}

\begin{frame}[fragile]
\frametitle{What are macros good for?}

\begin{itemize}
\item Code generation
\item Language virtualization
\item Type computations
\item Compile-time checks
\end{itemize}
\end{frame}

\begin{frame}[fragile]
\frametitle{Macros vs textual code generation}

Highlights:
\begin{itemize}
\item Structured (macros work with ASTs)
\item Type-aware (macros integrate with the typechecker)
\item Reflective (macros can reflect against the program being compiled)
\end{itemize}

\pause
\vskip15pt
Limitations:
\begin{itemize}
\item Only hardcore (macros 1.0 are really cumbersome)
\item Only expressions (macros 1.0 only include def macros)
\item Only local (macros 1.0 cannot make global changes to the program)
\item Only transient (macros 1.0 cannot generate code for humans)
\end{itemize}
\end{frame}

\begin{frame}[fragile]
\frametitle{Why am I highlighting the ``1.0'' part?}

\begin{itemize}
\item Because macros are rapidly evolving
\item In part thanks to external contributors like you!
\item A lot of cool things have been implemented after the 2.10.0 release
\item Which makes a lot of problems and restrictions go away
\item How? Now we're going to find out!
\end{itemize}
\end{frame}

\begin{frame}[fragile]
\frametitle{}

\vskip40pt
\begin{center}
\text{\color{blue}{\Large{Macros in paradise}}}
\end{center}
\end{frame}

\begin{frame}[fragile]
\frametitle{Macro paradise}

\begin{itemize}
\item An experimental fork of \texttt{scalac}, available for 2.10.x and 2.11.0:
\text{\color{blue}\href{http://docs.scala-lang.org/overviews/macros/paradise.html}{http://docs.scala-lang.org/overviews/macros/paradise.html}}
\item Compatible with the latest releases, i.e. with 2.10.2 and 2.11.0-M3
(this means you can use the libraries published for those releases!)
\item Nightlies are published to Sonatype and are easily accessible in SBT:
\begin{semiverbatim}
scalaVersion := "2.11.0-SNAPSHOT" or "2.10.2-SNAPSHOT"
scalaOrganization := "org.scala-lang.macro-paradise"
resolvers += Resolver.sonatypeRepo("snapshots")
\end{semiverbatim}
\end{itemize}
\end{frame}

\begin{frame}[fragile]
\frametitle{Cool new features}

\begin{itemize}
\item \text{\color{blue}{Quasiquotes (Denys Shabalin)}}
\item \text{\color{blue}{Implicit macros}}
\item \text{\color{blue}{Type macros}}
\item Macro annotations
\item Untyped macros
\item JIT compilation (Oleg Biruk)
\item Relaxed macros
\end{itemize}
\end{frame}

\begin{frame}[fragile]
\frametitle{Quasiquotes}

\begin{semiverbatim}
// tree manipulation 1.0
reify(List[T](element.splice))


                          \arrowdown

// tree manipulation 2.0
q"List[\$T](\$element)"
\end{semiverbatim}

\end{frame}

\begin{frame}[fragile]
\frametitle{Untyped snippets}

\begin{semiverbatim}
val fieldMemberType: Type = ...
reify \{
  new TypeBuilder \{
    type FieldType = fieldMemberType.splice // error!
  \}
\}

                          \arrowdown


q"new TypeBuilder \{ type FieldType = \$fieldMemberType \}"

\end{semiverbatim}

\begin{itemize}
\item Unlike \texttt{reify}, quasiquotes don't require their snippets to be typed
\item From experience, this is a vital feature for a metaprogramming system
\end{itemize}
\end{frame}

\begin{frame}[fragile]
\frametitle{Better splicing}

\begin{semiverbatim}
def foo(xs: Any*) = ...
val args: List[Expr[Any]] = ...
reify \{ foo(args.splice) \} // error!


                          \arrowdown


def foo(xs: Any*) = ...
q"foo(..\$args)"

\end{semiverbatim}

\begin{itemize}
\item \texttt{reify} supports splicing single strongly-typed trees and types
\item Quasiquotes allow splicing virtually anything anywhere it makes sense
\end{itemize}
\end{frame}

\begin{frame}[fragile]
\frametitle{Pattern matching}

\begin{semiverbatim}
expr match \{
  case reify(foo.splice(x.splice)) => x // error!
\}


                          \arrowdown


expr match \{
  case q"\$foo(\$x)" => x
\}

\end{semiverbatim}

\begin{itemize}
\item Being strongly-typed, \texttt{reify} is hard to marry with destructuring
\item Quasiquotes can pattern match in arbitrary positions in snippets
\end{itemize}
\end{frame}

\begin{frame}[fragile]
\frametitle{Implicit macros}

\begin{semiverbatim}
trait Reads[T] \{
  def reads(json: JsValue): JsResult[T]
\}

object Json \{
  def fromJson[T](json: JsValue)
    (implicit fjs: Reads[T]): JsResult[T]
\}

\end{semiverbatim}

\begin{itemize}
\item Type classes are an idiomatic way of writing extensible code in Scala
\item This is an example of typeclass-based design in Play
\end{itemize}
\end{frame}

\begin{frame}[fragile]
\frametitle{Implicit macros}

\begin{semiverbatim}
def fromJson[T](json: JsValue)
  (implicit fjs: Reads[T]): JsResult[T]

implicit val IntReads = new Reads[Int] \{
  def reads(json: JsValue): JsResult[T] = ...
\}

fromJson[Int](json) // you write
fromJson[Int](json)(IntReads) // you get

\end{semiverbatim}

\begin{itemize}
\item With type classes we externalize the moving parts
\item And then specify them elsewhere
\item Instances of type classes are provided once
\item And then \texttt{scalac} fills them in automatically
\end{itemize}
\end{frame}

\begin{frame}[fragile]
\frametitle{Before macros}

\begin{semiverbatim}
case class Person(name: String, age: Int)

implicit val personReads = (
  (__ \\ 'name).reads[String] and
  (__ \\ 'age).reads[Int]
)(Person)

\end{semiverbatim}

\begin{itemize}
\item Everything is done manually, hence boilerplate
\item There are alternatives, but they have downsides
\end{itemize}
\end{frame}

\begin{frame}[fragile]
\frametitle{Vanilla macros (2.10.0)}

\begin{semiverbatim}
implicit val personReads = Json.reads[Person]

\end{semiverbatim}

\begin{itemize}
\item Boilerplate can be generated by a macro
\item The code ends up being the same as if it were written manually
\end{itemize}
\end{frame}

\begin{frame}[fragile]
\frametitle{Implicit macros (2.10.2+)}

\begin{semiverbatim}
// no code necessary

\end{semiverbatim}

\begin{itemize}
\item Implicit values can be synthesized on-the-fly by a macro
\item Used with great success in scala-pickling
\item More information in \text{\color{blue}\href{http://www.meetup.com/Bay-Area-Scala-Enthusiasts/events/121848382/}{my tomorrow's talk in San Francisco}}
\end{itemize}
\end{frame}

\begin{frame}[fragile]
\frametitle{Type macros}
\begin{semiverbatim}
val brazilian = Db.Coffees.insert("Brazilian", 99, 0)
Db.Coffees.update(brazilian.copy(price = 10))
println(Db.Coffees.all)

\end{semiverbatim}

\begin{itemize}
\item Term macros can generate terms, type macros generate types
\item Imagine we need to create a strongly-typed wrapper for a database
\item Type macros are a great solution for that!
\end{itemize}
\end{frame}

\begin{frame}[fragile]
\frametitle{Type macros}
\begin{semiverbatim}
object Db extends H2Db("Coffees")\only<1>{






}\only<2->{

trait H2Db_Coffees \{
  class Coffee \{ ... \}
  val Coffees: Table[Coffee] = ...
\}
object Db extends H2Db_Coffees}

\end{semiverbatim}
\begin{itemize}
\item The \texttt{H2Db} macro takes a connection string
\only<1>{\item ...}
\only<2>{\item Then connects to the database and generates the wrapper}
\only<2>{\item Similar to type providers in F\#}
\end{itemize}
\end{frame}

\begin{frame}[fragile]
\frametitle{Type macros}
\begin{semiverbatim}
type H2Db(url: String) = macro impl\only<1>{






}\only<2->{

def impl(c: Context)(url: c.Tree) = \{
  val wrapper = q"trait Wrapper \{ \$\{generateCode(url)\} \}"\only<2>{
  ...}\only<3->{
  val wrapperRef = c.introduceTopLevel(wrappersPkg, wrapper)}\only<3>{
  ...}\only<4->{
  q"\$wrapperRef(\$url)"}
\}}

\end{semiverbatim}

\begin{itemize}
\only<1>{\item Definition and usage of type macros are the same as for def macros}
\only<1>{\item We start with a macro def and write its signature}
\only<2>{\item Now we proceed with the implementation}
\only<2->{\item The implementation creates a trait that encapsulates a database}
\only<3->{\item And then makes the newly created trait visible to the entire program}
\only<4->{\item Afterwards it expands into a reference to the wrapper}
\end{itemize}
\end{frame}

\begin{frame}[fragile]
\frametitle{Summary}

\begin{itemize}
\item Macro paradise hosts a lot of cool new features
\item Immediately available from Sonatype
\item Macro paradise is not a thing in itself, it targets upstream Scala
\item The most successful paradise features have already made it into Scala
\item Which ones? We'll see in a few minutes!
\end{itemize}
\end{frame}

\begin{frame}[fragile]
\frametitle{}

\vskip40pt
\begin{center}
\text{\color{blue}{\Large{The future of macros}}}
\end{center}
\end{frame}

\begin{frame}[fragile]
\frametitle{Macros 1.0 are great}

\begin{itemize}
\item Things that were previously impossible are now within reach
\begin{itemize}
\item People are using macros to bring their ideas to life
\item Typesafe employs macros in a number of projects
\item At LAMP we are using macros to power our research
\end{itemize}
\end{itemize}
\end{frame}

\begin{frame}[fragile]
\frametitle{Macros 1.0 are complicated}

\begin{itemize}
\item Annoying
\begin{itemize}
\item Hard to grasp
\item Hard to use
\end{itemize}
\vskip15pt
\item Volatile
\begin{itemize}
\item A lot of freedom type-wise
\item A lot of freedom execution-wise
\end{itemize}
\end{itemize}
\end{frame}

\begin{frame}[fragile]
\frametitle{The macro conundrum}

\begin{itemize}
\item Macros 1.0 are annoying
\item Macros 1.0 are volatile
\item But we still want macros, because they are so great!
\end{itemize}
\end{frame}

\begin{frame}[fragile]
\frametitle{Macros 2.0}

\begin{itemize}
\pause
\item Simplify
\begin{itemize}
\item Quasiquotes!
\item The rest of reflection API
\item Better IDE support (debugging, inline expansion, Intellij)
\end{itemize}
\pause
\vskip15pt
\item Stratify
\begin{itemize}
\item Codify the conservative ones (stable subset)
\item Let the powerful ones evolve (experimental subset)
\end{itemize}
\end{itemize}
\end{frame}

\begin{frame}[fragile]
\frametitle{How does one stratify macros?}

\begin{itemize}
\item By answering a simple question
\begin{itemize}
\item Do we have to expand this macro to typecheck the program?
\end{itemize}

\vskip20pt
\item This is quite equivalent to the questions
\begin{itemize}
\item Does a human have to expand this macro to understand the program?
\item Does an IDE have to expand this macro to analyze the program?
\item Does this macro really taste like a method?
\end{itemize}
\end{itemize}
\end{frame}

\begin{frame}[fragile]
\frametitle{Blackbox macros}

\begin{itemize}
\item The conservative ones
\item Don't affect typechecking
\item One can say they are opaque to the typer, hence the name
\item \texttt{BlackboxContext} = quasiquotes + just a bit more
\end{itemize}
\end{frame}

\begin{frame}[fragile]
\frametitle{Whitebox macros}

\begin{itemize}
\item The powerful ones
\item For them everything stays as it is now and will continue evolving
\item \texttt{WhiteboxContext} = \texttt{Context} of macros 1.0 + later developments
\end{itemize}
\end{frame}

\begin{frame}[fragile]
\frametitle{Summary}

\begin{itemize}
\item Our primary goal for now is to make macros easy to use
\item Then we plan to bring blackbox macros into the language
\item Are blackbox macros good enough? Time will tell
\item In the meanwhile we will still be experimenting with whitebox macros
\end{itemize}
\end{frame}

\begin{frame}[fragile]
\frametitle{}

\vskip40pt
\begin{center}
\text{\color{blue}{\Large{\href{http://groups.google.com/group/scala-internals/browse_thread/thread/f755b4f8fc4c43d4}{The roadmap for macros in Scala 2.10+}}}}
\end{center}
\end{frame}

\begin{frame}[fragile]
\frametitle{2.10.x}

Experimental:
\begin{itemize}
\item Reflection (2.10.0+, not going anywhere)
\item Macros 1.0 (2.10.0+, not going anywhere)
\item Implicit macros (2.10.2+, single-parametric type classes only)
\item Quasiquotes (2.10.0+, \text{\color{blue}\href{http://docs.scala-lang.org/overviews/macros/paradise.html\#macro_paradise_for_210x}{quasi-supported}} via paradise 2.10.x)
\end{itemize}
\end{frame}

\begin{frame}[fragile]
\frametitle{2.11.0}

Experimental (looking good for becoming stable in 2.12):
\begin{itemize}
\item Blackbox macros
\item Quasiquotes
\item Macro bundles
\end{itemize}

\vskip20pt
Experimental (needing more time for evaluation):
\begin{itemize}
\item Reflection
\item Whitebox macros
\item Implicit macros (single-parametric type classes only)
\item asInstanceOf[scala.reflect.internal.SymbolTable]
\end{itemize}
\end{frame}

\begin{frame}[fragile]
\frametitle{Paradise}

Look good for promotion to 2.11.0, but need time that we might not have before the release:
\begin{itemize}
\item Implicit macros (multi-parametric type classes)
\item Macro annotations
\end{itemize}

\vskip20pt
Won't be promoted to 2.11.0, ordered by descending likelihood of making it into any Scala at all:
\begin{itemize}
\item \texttt{introduceTopLevel}
\item Untyped macros
\item Type macros
\end{itemize}
\end{frame}

\begin{frame}[fragile]
\frametitle{Summary}

\begin{itemize}
\item Macros are here to stay
\item Blackbox macros are going to be stabilized in 2.12
\item But whitebox macros will still stick around as experimental
\item So your macros will continue working in 2.11 and probably in 2.12
\item Type macros didn't make it, macro annotations will take their place
\end{itemize}
\end{frame}

\begin{frame}[fragile]
\frametitle{}

\vskip40pt
\begin{center}
\text{\color{blue}{\Large{Wrapping up}}}
\end{center}
\end{frame}

\begin{frame}[fragile]
\frametitle{Summary}

\begin{itemize}
\item Macros 1.0 are popular among production and research users of Scala
\item We created a fork of \texttt{scalac} called macro paradise
\item In paradise we have been experimenting with our design
\item And we came up with a bunch of improvements for macros 1.0
\item This will make macros easy to use and accessible for everyone
\end{itemize}
\end{frame}

\begin{frame}[fragile]
\frametitle{Or in other words}

\begin{itemize}
\item Macros were created by man
\item They rebelled
\item They evolved
\item There are many flavors
\item And they have a plan
\end{itemize}
\end{frame}

\end{document}
