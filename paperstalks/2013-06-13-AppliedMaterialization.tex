\documentclass[svgnames,hyperref={bookmarks=false}]{beamer}
\useoutertheme{infolines}
\setbeamertemplate{headline}{} % removes the headline that infolines inserts
% \setbeamertemplate{footline}{
%   \hfill%
%   \usebeamercolor[fg]{page number in head/foot}%
%   \usebeamerfont{page number in head/foot}%
%   \insertpagenumber\,/\,\insertpresentationendpage\kern1em\vskip2pt%
% }
\setbeamertemplate{footline}{
  \hfill%
  \usebeamercolor[fg]{page number in head/foot}%
  \usebeamerfont{page number in head/foot}%
  \footnotesize\insertpagenumber\kern1em\vskip2pt%
}
\setbeamertemplate{navigation symbols}{}
\setbeamercolor{alerted text}{fg=blue}
\setbeamerfont{alerted text}{series=\bfseries,family=\ttfamily}
\setbeamertemplate{section in toc}{\text{\color{black}{\inserttocsection}}}
\usepackage[parfill]{parskip}
\usepackage[linewidth=0.5pt]{mdframed}
\newmdenv[innerleftmargin=1mm, innerrightmargin=1mm, innertopmargin=-1mm, innerbottommargin=2mm, leftmargin=-1mm, rightmargin=-1mm]{lstlistinglike}
\usepackage{tikz}
\usepackage{graphicx}
\DeclareGraphicsExtensions{.pdf,.png,.jpg}
\usepackage{textcomp}
\usepackage{pifont}
\usepackage{tabu}
\usepackage{fixltx2e}
\usetikzlibrary{shapes.arrows}

\tikzset{
    myarrow/.style={
        draw,
        fill=orange,
        single arrow,
        minimum height=3.5ex,
        single arrow head extend=1ex
    }
}
\newcommand{\arrowup}{%
\tikz [baseline=-0.5ex]{\node [myarrow,rotate=90] {};}
}
\newcommand{\arrowdown}{%
\tikz [baseline=-1ex]{\node [myarrow,rotate=-90] {};}
}

\hypersetup{pdfauthor={Eugene Burmako},pdfsubject={Applied Materialization: When Macros Meet Implicits},pdftitle={Applied Materialization: When Macros Meet Implicits}}
\title{Applied Materialization: When Macros Meet Implicits}

\begin{document}

\title{Applied Materialization}
\subtitle{When Macros Meet Implicits}
\author{Eugene Burmako}
\institute{\'Ecole Polytechnique F\'ed\'erale de Lausanne \\
           \texttt{http://scalamacros.org/}}
\date{13 June 2013}
{
\setbeamertemplate{footline}{}
\begin{frame}
  \titlepage
\end{frame}
}

\begin{frame}[fragile]
\frametitle{Agenda}

\text{\color{blue}\href{http://scalamacros.org/paperstalks/2013-06-12-HalfYearInMacroParadise.pdf}{My yesterday's talk at ScalaDays}} was about:
\begin{itemize}
\item New developments in macros after 2.10.0
\item Reflection on our experience with macros
\item The future of macros in Scala 2.10+
\end{itemize}

\vskip20pt
Today we will:
\begin{itemize}
\item Take on the challenge of generic programming
\item Code up a couple of macros using new features from macro paradise
\item Discuss materialization, my favorite application of macros
\end{itemize}
\end{frame}

\begin{frame}[fragile]
\frametitle{}

\vskip40pt
\begin{center}
\text{\color{blue}{\Large{Setting the stage}}}
\end{center}
\end{frame}

\begin{frame}[fragile]
\frametitle{Our running example}

\begin{semiverbatim}
trait Pickler[T] \{
  def pickle(x: T): Pickle
\}

def toPickle[T: Pickler](x: T): Pickle = \{
  implicitly[Pickler[T]].pickle(x)
\}

\end{semiverbatim}

\begin{itemize}
\item This is an example of typeclass-based design
\item Type classes are an idiomatic way of writing extensible code in Scala
\item But let's leave off the type class discussion until \text{\color{blue}\href{http://www.meetup.com/Bay-Area-Scala-Enthusiasts/events/121849802/}{Vlad's talk in July}}
\end{itemize}
\end{frame}

\begin{frame}[fragile]
\frametitle{How it works - part 1}

\begin{semiverbatim}
def toPickle[T: Pickler](x: T) = \{
  implicitly[Pickler[T]].pickle(x)
\}

def toPickle[T](x: T)(implicit evidence\$1: Pickler[T]) = \{
  implicitly[Pickler[T]](evidence\$1).pickle(x)
\}

\end{semiverbatim}

\begin{itemize}
\item The context bound is desugared into an implicit parameter
\item The call to \texttt{implicitly} summons that implicit and calls it to action
\end{itemize}
\end{frame}

\begin{frame}[fragile]
\frametitle{How it works - part 2}

\begin{semiverbatim}
def toPickle[T](x: T)(implicit p: Pickler[T]): Pickle = ...

implicit object IntPickler extends Pickler[Int] \{ ... \}

toPickle(42) // toPickle(42)(IntPickler)
toPickle("42") // compile-time error

\end{semiverbatim}

\begin{itemize}
\item To use \texttt{toPickle} we declared implicit picklers in scope
\item The compiler is then smart enough to figure them out
\end{itemize}
\end{frame}

\begin{frame}[fragile]
\frametitle{How it works - part 3}

\begin{semiverbatim}
def toPickle[T](x: T)(implicit p: Pickler[T]): Pickle = ...

object IntPickler extends Pickler[Int] \{ ... \}
implicit def listPickler[T: Pickler]: List[Pickler[T]] = ...

toPickle(List(42)) // toPickle(...)(listPickler(IntPickler))

\end{semiverbatim}

\begin{itemize}
\item It gets even better
\item For instance, implicits are composable
\item Here a \texttt{List[T]} pickler gets built up from a pickler for \texttt{T}
\end{itemize}
\end{frame}

\begin{frame}[fragile]
\frametitle{How it works - part 4}

\begin{itemize}
\item The strength of the typeclass-based design lies in its flexibility
\item Implicits are composable
\item Implicits can be scoped
\item Implicits allow for unanticipated evolution
\item Though again I'm getting ahead of the \text{\color{blue}\href{http://www.meetup.com/Bay-Area-Scala-Enthusiasts/events/121849802/}{upcoming July talk}}
\end{itemize}
\end{frame}

\begin{frame}[fragile]
\frametitle{The problem statement}

\begin{semiverbatim}
def toPickle[T: Pickler](x: T): Pickle = ...
case class Person(name: String)
toPickle(Person("Eugene"))

\end{semiverbatim}

\begin{itemize}
\item How do we make this code snippet compile?
\item Is our approach going to be scalable?
\end{itemize}
\end{frame}

\begin{frame}[fragile]
\frametitle{}

\vskip40pt
\begin{center}
\text{\color{blue}{\Large{A closer look}}}
\end{center}
\end{frame}

\begin{frame}[fragile]
\frametitle{Straightforward approach}

\begin{semiverbatim}
case class Person(name: String)

implicit val personPickler = new Pickler[Person] \{
  def pickle(x: Person): Pickle = \{
    emptyPickle + toPickle(x.name)
  \}
\}

\end{semiverbatim}

\begin{itemize}
\item Straightforward and obvious
\item But what if we have some more classes?
\item Do we copy/paste the code changing names and adding more fields?
% \item What are we, savages? \copyright Paul Phillips
\item We need some technique to abstract the tedium away
\end{itemize}
\end{frame}

\begin{frame}[fragile]
\frametitle{Reflective approach}

\begin{semiverbatim}
implicit def genericPickler[T: TypeTag]: Pickler[T] =
  new Pickler[T] \{ def pickle(x: T) = reflect(x) \}

def reflect(x: T): Pickle = \{
  val fields = typeOf[T].declarations.collect \{
    case sym: MethodSymbol if sym.isGetter => sym \}
  val m = currentMirror.reflect(x)

  fields.foldLeft(emptyPickle)((p, f) => \{
    p + \only<1>{toPickle(m.reflectMethod(g)())}\only<2>{\alert{toPickle(}m.reflectMethod(g)()\alert{: Any)}}
  \})
\}
\end{semiverbatim}

\begin{itemize}
\item Properly generalizes over case classes, no client code required at all
\item But has subpar performance
\item And is not type-safe\only<1>{ (can you spot the bug?)}\only<2>{, because the reflective invocation returns \texttt{Any}}
\end{itemize}
\end{frame}

\begin{frame}[fragile]
\frametitle{Typelevel approach}

\begin{itemize}
\item The good thing about reflection is that it can treat data uniformly
\item The bad thing is that the representation it uses is dynamically typed
\item Luckily there exists a statically typed solution
\item Enter \texttt{HLists}
\end{itemize}
\end{frame}

\begin{frame}[fragile]
\frametitle{Typelevel approach}

\begin{semiverbatim}
case class Apple() extends Fruit
case class Pear() extend Fruit

val a: Apple = Apple()
val p: Pear = Pear()
\only<1>{}\only<2>{
type APAP = Apple :: Pear :: Apple :: Pear :: HNil}
val hlist\only<2>{\alert{: APAP}} = a :: p :: a :: p :: HNil
val list\only<2>{\alert{: List[Fruit]}} = a :: p :: a :: p :: Nil

\end{semiverbatim}
\onslide

\begin{itemize}
\item On the surface \texttt{HList} is quite similar to \texttt{List}
\pause
\item But it is much more precise type-wise
\item Yesterday at ScalaDays \text{\color{blue}\href{http://www.scaladays.org/\#/june-12/room3/14:30-15:15/Expanding-eta-expansion\%3A-shapeless-polymorphic-function-values-meet-macros}{Miles}} worked magic enabled by \texttt{HLists}
\item And \text{\color{blue}\href{???}{Alois}} brought \texttt{HLists} even further by adding labels
\end{itemize}
\end{frame}

\begin{frame}[fragile]
\frametitle{Typelevel approach}

\begin{semiverbatim}
trait Generic[T, R] \{
  def to(t: T): R
  def from(r: R): T
\}

implicit val persIso = new Generic[Person, String :: HNil] \{
  def to(t: Person) = t.name :: HNil
  def from(r: String :: HNil) = Person(r.head)
\}

\end{semiverbatim}

\begin{itemize}
\item After the uniform representation for data is picked
\item For every data type we define an isomorphism to the repr
\end{itemize}
\end{frame}

\begin{frame}[fragile]
\frametitle{Typelevel approach}

\begin{semiverbatim}
implicit val hnilPickler: Pickler[HNil] =
  new Pickler[HNil] \{ def pickle(x: HNil) = emptyPickle \}

implicit def hlistPickler[H, T <: HList]
  (implicit ph: Pickler[H],
            pt: Pickler[T]): Pickler[H :: T] = \{
  new Pickler[H :: T] \{
    def pickle(x: H :: T) =
      ph.pickle(x.head) + pt.pickle(x.tail)
  \}
\}

\end{semiverbatim}

\begin{itemize}
\item Now the compiler knows how to treat our data types uniformly
\item Therefore a single serializer for repr will make all our data serializable
\item This is type-safe and overall cool, but still not very performant
\end{itemize}
\end{frame}

\begin{frame}[fragile]
\frametitle{A detour: the bootstrapping challenge}

\begin{semiverbatim}
implicit val persIso = new Generic[Person, String :: HNil] \{
  def to(t: Person) = t.name :: HNil
  def from(r: String :: HNil) = Person(r.head)
\}

\end{semiverbatim}

\begin{itemize}
\item To serialize data types generically, we need to isomorphize them
\item But isomophization itself is a generic programming task!
\item Which comes first, the chicken or the egg?
\pause
\begin{itemize}
\item The chicken
\item Isomophization can be treated differently from the other GP problems
\item Once somehow solved, it will take care of everything else
\end{itemize}
\end{itemize}
\end{frame}

\begin{frame}[fragile]
\frametitle{Summary of generic programming techniques}

{\tabulinesep=1.5mm
\begin{tabu}{ | p{2cm} | p{3cm} | p{2.5cm} | p{2.5cm} | }
  \hline
  Technique & \parbox{3cm}{Client-side \\ convenience\textsuperscript{1}} & Performance\textsuperscript{2} & \parbox{2.5cm}{Library-side \\ convenience\textsuperscript{3}} \\ \hline
  Manual\textsuperscript{*} & Bad & Excellent & Excellent \\ \hline
  Reflection & Excellent & Bad & Decent \\ \hline
  Typelevel & Good & Good & Good \\ \hline
\end{tabu}
}

\textsuperscript{*} Not a generic programming technique, is here just for comparison\\
\textsuperscript{1} How much effort is required to add support for a new data type?\\
\textsuperscript{2} How does the performance fare against manually written code?\\
\textsuperscript{3} How much effort is required from a library author?\\
\end{frame}

\begin{frame}[fragile]
\frametitle{}

\vskip40pt
\begin{center}
\text{\color{blue}{\Large{Level 1: def macros}}}
\end{center}
\end{frame}

\begin{frame}[fragile]
\frametitle{Macro-based approach}

\begin{semiverbatim}
implicit val personPickler = new Pickler[Person] \{
  def pickle(x: Person): Pickle = \{
    emptyPickle + toPickle(x.name)
  \}
\}

\end{semiverbatim}

\begin{itemize}
\item Temporarily back to square one
\item We will start with the simplest thing possible
\item And will make it enjoyable to use
\end{itemize}
\end{frame}

\begin{frame}[fragile]
\frametitle{Macro-based approach}

\begin{semiverbatim}
implicit val personPickler = \alert{Pickler.genericPickler[Person]}




\end{semiverbatim}

\begin{itemize}
\item Def macros expand method calls into code blocks
\item Expansion happens at compile-time when compiler sees a macro call
\item When invoked, macros programmatically construct their expansion
\item Arguments of a macro call are available via the reflection API
\item Therefore we can write a macro to generate the body of the implicit
\end{itemize}
\end{frame}

\begin{frame}[fragile]
\frametitle{Let's write a macro}

\begin{itemize}
\item Compile-time reflection has the same API as runtime reflection
\item With the newly introduced quasiquotes code generation is a breeze
\item Therefore we can easily turn our reflective pickler into a macro!
\item For details tag along or follow
\text{\color{blue}\href{http://docs.scala-lang.org/overviews/}{our documentation}}
\end{itemize}
\end{frame}

\begin{frame}[fragile]
\frametitle{Before we begin}

\begin{itemize}
\item Some of the features we are going to use aren't yet in Scala 2.10
\item Those features come from macro paradise, an experimental fork of \texttt{scalac}, available for 2.10.x and 2.11.0
(details at \text{\color{blue}\href{http://docs.scala-lang.org/overviews/macros/paradise.html}{docs.scala-lang.org}})
\item A lot of new developments from paradise are going to end up in 2.11.0
\item When introducing features, I will be mentioning their Scala versions
\end{itemize}
\end{frame}

\begin{frame}[t, fragile]
\frametitle{Step 1: Start with a reflective pickler}

\begin{semiverbatim}


import scala.reflect.runtime.universe.\_
object Pickler \{
  implicit def genericPickler[T: TypeTag]: Pickler[T] = \{
    val T = typeOf[T]
    val fields = T.declarations.collect \{ ... \}
    def reflect(x: T): Pickle = ...
    new Pickler[T] \{ def pickle(x: T) = reflect(x) \}
  \}
\}
\end{semiverbatim}
\end{frame}

\begin{frame}[t, fragile]
\frametitle{Step 2: Rebrand it as a macro}

\begin{semiverbatim}
\alert{def genericPickler[T]: Pickler[T] =
  macro PicklerMacro.genericPickler[T]}

\alert{trait PicklerMacro extends scala.reflect.macros.Macro} \{
  \text{\color{black}{def genericPickler[T: TypeTag]: Pickler[T] = \{}}
    \text{\color{black}{val T = typeOf[T]}}
    \text{\color{black}{val fields = T.declarations.collect \{ ... \}}}
    \text{\color{black}{def reflect(x: T): Pickle = ...}}
    \text{\color{black}{new Pickler[T] \{ def pickle(x: T) = reflect(x) \}}}
  \text{\color{black}{\}}}
\}
\end{semiverbatim}
\end{frame}

% \begin{frame}[t, fragile]
% \frametitle{Step 2: Rebrand it as a macro}

% \begin{semiverbatim}
% \alert{def genericPickler[T]: Pickler[T] =
%   macro PicklerMacro.genericPickler[T]}

% trait PicklerMacro extends scala.reflect.macros.Macro \{
%   ...
%   ...
%   ...
%   ...
%   ...
%   ...
% \}
% \end{semiverbatim}
% \end{frame}

\begin{frame}[t, fragile]
\frametitle{Step 3: Make it produce trees}

\begin{semiverbatim}
def genericPickler[T]: Pickler[T] =
  macro PicklerMacro.genericPickler[T]

trait PicklerMacro extends scala.reflect.macros.Macro \{
  def genericPickler[T: WeakTypeTag]: \alert{Tree} = \{
    val T = weakTypeOf[T]
    val fields = T.declarations.collect \{ ... \}
    def reflect: \alert{Tree} = ...
    \alert{q"}new Pickler[\alert{\$}T] \{ def pickle(x: \alert{\$}T) = \alert{\$}reflect \}\alert{"}
  \}
\}

\end{semiverbatim}

\begin{itemize}
\item This macro can be written in Scala 2.10.0, yet in a very verbose way
\item However here we use quasiquotes planned for 2.11.0-M4 (8 Jul 2013)
\item Those who wrote macros in 2.10, note how easy it is to do it now!
\end{itemize}
\end{frame}

\begin{frame}[fragile]
\frametitle{Step 4: Enjoy!}

In the source file you write:
\begin{semiverbatim}
implicit val personPickler = Pickler.genericPickler[Person]

\end{semiverbatim}

Under the covers it becomes:
\begin{semiverbatim}
implicit val personPickler = new Pickler[Person] \{
  def pickle(x: Person): Pickle = \{
    emptyPickle + toPickle(x.name)
  \}
\}
\end{semiverbatim}
\end{frame}

% \begin{frame}[fragile]
% \frametitle{Some remarks}

% \begin{itemize}
% \item This code should work with 2.11.0-final
% \item Those wrote wrote macros in 2.10, note how easy it is to do it now!
% \item Quasiquotes are already magic, but are not yet hygienic
% \begin{itemize}
% \item Therefore things like \texttt{q"Pickler"} need to be fully qualified
% \item Like that: \texttt{q"scala.pickling.Pickler"}
% \item But not like that: \texttt{q"import scala.pickling.\_"}
% \item Because in the latter case \texttt{Pickler} might be hijacked by a local
% \end{itemize}
% \end{itemize}
% \end{frame}

\begin{frame}[fragile]
\frametitle{Summary of generic programming techniques}

{\tabulinesep=1.5mm
\begin{tabu}{ | p{2cm} | p{3cm} | p{2.5cm} | p{2.5cm} | }
  \hline
  Technique & \parbox{3cm}{Client-side \\ convenience} & Performance & \parbox{2.5cm}{Library-side \\ convenience} \\ \hline
  Manual & Bad & Excellent & Excellent \\ \hline
  Reflection & Excellent & Bad & Decent \\ \hline
  Typelevel & Good & Good & Good \\ \hline
  \text{\color{blue}{Macros\textsuperscript{\textdagger}}} & \text{\color{blue}{Good}} & \text{\color{blue}{Excellent}} & \text{\color{blue}{Decent}} \\
  \hline
\end{tabu}}

\onslide
\text{\color{blue}{\textsuperscript{\textdagger} Requires def macros (Scala 2.10.0+)}}
\end{frame}

\begin{frame}[fragile]
\frametitle{}

\vskip40pt
\begin{center}
\text{\color{blue}{\Large{Level 2: materialization}}}
\end{center}
\end{frame}

\begin{frame}[fragile]
\frametitle{A detour: synergy}

In Scala, macros work in harmony with rich syntax and static types:
\begin{itemize}
\item String interpolation + macros
\item Implicits + macros
\item Types + macros
\item Annotations + macros
\end{itemize}

\vskip20pt
More on that in my recent paper:
\text{\color{blue}\href{http://scalamacros.org/paperstalks/2013-04-22-LetOurPowersCombine.pdf}{"Scala Macros: Let Our Powers Combine!"}}
\end{frame}

\begin{frame}[fragile]
\frametitle{Revisiting our current solution}

\begin{semiverbatim}
def genericPickler[T]: Pickler[T] = macro ...
implicit val personPickler = Pickler.genericPickler[Person]
implicit val repoPickler = Pickler.genericPickler[Repo]
implicit val commitPickler = Pickler.genericPickler[Commit]
...

\end{semiverbatim}

\begin{itemize}
\item One generic implementation
\item One line of code per data type
\end{itemize}
\end{frame}

\begin{frame}[fragile]
\frametitle{Implicit macros}

\begin{semiverbatim}
\alert{implicit} def genericPickler[T]: Pickler[T] = macro ...





\end{semiverbatim}

\begin{itemize}
\item One generic implementation
\item \alert{Zero} lines of code per data type
\item When a \texttt{Pickler} is missing, one is generated on the fly
\item This is a new feature in Scala 2.10.2+
\end{itemize}
\end{frame}

\begin{frame}[fragile]
\frametitle{How it works - part 1}

\begin{semiverbatim}
trait Pickler[T] \{ def pickle(x: T): Pickle \}

object Pickler \{
  implicit def genericPickler[T]: Pickler[T] = macro ...
\}

\end{semiverbatim}

\begin{itemize}
\item When \texttt{scalac} looks for implicits, it traverses the implicit scope
\item Implicit scope transcends lexical scope
\item Among others it includes members of the target's companion
\end{itemize}
\end{frame}

\begin{frame}[fragile]
\frametitle{How it works - part 2}

\begin{semiverbatim}
\alert{implicit} def genericPickler[T]: Pickler[T] = macro ...

trait PicklerMacro extends Macro \{
  def genericPickler[T: WeakTypeTag]: Tree = \{
    ...
    q"new Pickler[\$T] \{ def pickle(x: \$T) = \$reflect \}"
  \}
\}

\end{semiverbatim}

\begin{itemize}
\item Here's our def macro from before
\item We have just made it implicit
\item Are we done yet? No!
\end{itemize}
\end{frame}

\begin{frame}[t, fragile]
\frametitle{How it works - part 2}

\begin{semiverbatim}
case class List(head: Int, tail: List)

val list: List = ...
toPickle(list)







\end{semiverbatim}

\begin{itemize}
\item To illustrate the caveat let's take a recursive data type
\item And see how its materializer is going to expand
\end{itemize}
\end{frame}

\begin{frame}[t, fragile]
\frametitle{How it works - part 2}

\begin{semiverbatim}
case class List(head: Int, tail: List)

val list: List = ...
toPickle(list)(\{
  new Pickler[List] \{
    def pickle(x: List) = \{
      emptyPickle +
      toPickle(x.head)\only<2>{(IntPickler)} +
      toPickle(x.tail)\only<2>{(\alert{new Pickler[List] \{ ... \}})}
    \}
  \}
\})
\end{semiverbatim}

\begin{itemize}
\item After the first expansion we get two recursive calls to \texttt{toPickle}
\item The first one will be resolved to \texttt{IntPickler}, that's easy
\item But what about the second one? \pause Uh-oh!
\end{itemize}
\end{frame}

\begin{frame}[t, fragile]
\frametitle{How it works - part 2}

\begin{semiverbatim}
case class List(head: Int, tail: List)

val list: List = ...
toPickle(\{
  \alert{implicit object ListPickler} extends Pickler[List] \{
    def pickle(x: List) = \{
      emptyPickle +
      toPickle(x.head)(IntPickler) +
      toPickle(x.tail)(\alert{ListPickler})
    \}
  \}
  ListPickler
\})
\end{semiverbatim}

\begin{itemize}
\item We also need to deal with possible recursion
\item And we do that by tying the knot using implicits themselves!
\end{itemize}
\end{frame}

\begin{frame}[t, fragile]
\frametitle{Nitpicking time!}

\begin{semiverbatim}
case class List(head: Int, tail: List)

val list: List = ...
toPickle(list)

\end{semiverbatim}

\begin{itemize}
\item Our design has just stood up to a serious test
\item But, in fact, this very example is spectacularly incomplete
\item It hints at design issues we haven't yet discussed. What are they?
\pause
\begin{itemize}
\item Are algebraic data types supported?
\item Can we declare \texttt{head} to be polymorphic?
\item If not polymorphic, can it be \texttt{Any}?
\end{itemize}
\end{itemize}

\vskip15pt
Heather's \text{\color{blue}\href{http://lampwww.epfl.ch/~hmiller/pickling/}{recent paper}} answers all these questions, and her
cool new \text{\color{blue}\href{http://github.com/scala/pickling}{scala-pickling project}} makes the dreams come true!
\end{frame}

\begin{frame}[fragile]
\frametitle{Summary of generic programming techniques}

{\tabulinesep=1.5mm
\begin{tabu}{ | p{2cm} | p{3cm} | p{2.5cm} | p{2.5cm} | }
  \hline
  Technique & \parbox{3cm}{Client-side \\ convenience} & Performance & \parbox{2.5cm}{Library-side \\ convenience} \\ \hline
  Manual & Bad & Excellent & Excellent \\ \hline
  Reflection & Excellent & Bad & Decent \\ \hline
  Typelevel & Good & Good & Good \\ \hline
  Macros\textsuperscript{\textdagger} & Good / \text{\color{blue}{Excellent\textsuperscript{$\ddagger$}}} & Excellent & Decent \\
  \hline
\end{tabu}
}

\textsuperscript{\textdagger} Requires def macros (Scala 2.10.0+)
\text{\color{blue}{\textsuperscript{$\ddagger$} Requires implicit macros (Scala 2.10.2+)}}
\end{frame}

\begin{frame}[fragile]
\frametitle{}

\vskip40pt
\begin{center}
\text{\color{blue}{\Large{Level 3: fundep materialization}}}
\end{center}
\end{frame}

\begin{frame}[fragile]
\frametitle{Isomorphisms}

\begin{semiverbatim}
trait Generic[T, R] \{
  def to(t: T): R
  def from(r: R): T
\}

\end{semiverbatim}

\begin{itemize}
\item Let's use our newly acquired proficiency in materialization
\item By writing a materializer for the isomorphisms mentioned earlier
\end{itemize}
\end{frame}

\begin{frame}[t, fragile]
\frametitle{Materializing isomorphisms}

\begin{semiverbatim}
implicit def materialize[T\only<2>{\alert{, R}}]: Generic[T\only<2>{\alert{, R}}] =
  macro GenericMacro.materialize[T]

trait GenericMacro extends Macro \{
  def materialize[T: WeakTypeTag]: Tree = \{
    val T = weakTypeOf[T]
    val R = calculateRepr(T)
    q"new Generic[\$T, \$R] \{ ... \}"
  \}
\}

\end{semiverbatim}

\begin{itemize}
\item Copy/pasting, adjusting - okay, so far so good
\item There is a mistake here. Where is it?
\pause
\begin{itemize}
\item We forgot the second type parameter of \texttt{Generic}
\end{itemize}
\end{itemize}
\end{frame}

\begin{frame}[t, fragile]
\frametitle{Using the materializer}

\begin{semiverbatim}
implicit def materialize[T, R]: Generic[T, R] =
  macro GenericMacro.materialize[T]

implicit def genericPickler[T, R]
  (implicit iso: Generic[T, R],
            p: Pickler[R]): Pickler[T] = \{
  new Pickler[T] \{
    def pickle(x: T) = p.pickle(iso.to(x))
  \}
\}

\end{semiverbatim}

\begin{itemize}
\item Double materialization!
\item Requests for a pickler for \texttt{T} will be processed by \texttt{genericPickler}
\item That will materialize \texttt{Generic[T, R]} which will figure out \texttt{R}
\item And that will materialize \texttt{Pickler[R]} that does the heavylifting
\end{itemize}
\end{frame}

\begin{frame}[t, fragile]
\only<1>{\frametitle{Problem \#1}}
\only<2>{\frametitle{Non-problem \#1}}

\begin{semiverbatim}
implicit val hnilPickler: Pickler[HNil] = ...

implicit def hlistPickler[H, T <: HList]
  (implicit ph: Pickler[H],
            pt: Pickler[T]): Pickler[H :: T] = ...

implicit def genericPickler[T, R]
  (implicit iso: Generic[T, R],
            p: Pickler[R]): Pickler[T] = ...

\end{semiverbatim}

\begin{itemize}
\item We have overlapping implicits
\item Is that going to be a problem?
\pause
\begin{itemize}
\item Not really
\item For any given \texttt{T}, \texttt{scalac} can figure out the most specific one
\end{itemize}
\end{itemize}
\end{frame}

\begin{frame}[fragile]
\frametitle{Problem \#2}

\begin{semiverbatim}
09:09 ~/Projects/210x \$ scalac Test.scala -Xlog-implicits
Test.scala:6: error: could not find implicit value
for parameter iso: Generic[Test.Foo,R]
  toPickle(Person("Eugene"))
          ^
\end{semiverbatim}

\begin{itemize}
\item The ever so helpful missing implicit message!
\item Let's figure out what went wrong with the help of \texttt{-Xlog-implicits}
\end{itemize}
\end{frame}

\begin{frame}[fragile]
\frametitle{Problem \#2}

\begin{semiverbatim}
09:09 ~/Projects/210x \$ scalac Test.scala -Xlog-implicits
Test.scala:13: materialize is not a valid implicit value
for Generic[Person, R] because:
hasMatchingSymbol reported error: type mismatch;
 found   : Generic[Person, String :: HNil]
 required: Generic[Person, Nothing]
  toPickle(Person("Eugene"))
          ^
\end{semiverbatim}

\begin{itemize}
\item What's going on? Where did the \texttt{Nothing} come from?
\end{itemize}
\end{frame}

\begin{frame}[fragile]
\frametitle{Problem \#2}

\begin{semiverbatim}
toPickle(Person("Eugene"))

                          \arrowdown

toPickle(Person("Eugene"))(materialize[?, ?])

\end{semiverbatim}

\begin{itemize}
\item When inferring the implicit argument for the call to \texttt{toPickle}
\item \texttt{scalac} finds \texttt{Generic.materialize} as a candidate
\item And then it needs to check whether the type parameters work out
\item The first is replacing all of them with unknowns
\end{itemize}
\end{frame}

\begin{frame}[fragile]
\frametitle{Problem \#2}

\begin{semiverbatim}
toPickle(Person("Eugene"))(materialize[?, ?])

                          \arrowdown

toPickle(Person("Eugene"))(materialize[Person, ?])

\end{semiverbatim}

\begin{itemize}
\item Using the information provided in the call to \texttt{toPickle}
\item \texttt{scalac} is able to infer \texttt{T} to \texttt{Person}
\item However \texttt{R} remains unknown, because nothing hints \texttt{scalac} about it
\end{itemize}
\end{frame}

\begin{frame}[fragile]
\frametitle{Problem \#2}

\begin{semiverbatim}
toPickle(Person("Eugene"))(materialize[Person, ?])

                          \arrowdown

toPickle(Person("Eugene"))(materialize[Person, Nothing])

\end{semiverbatim}

\begin{itemize}
\item Macros cannot expand with uninferred type arguments
\item Therefore \texttt{scalac} has to go to extreme measures
\item Inferring \texttt{R} to a default, which is in this case \texttt{Nothing}
\end{itemize}
\end{frame}

\begin{frame}[fragile]
\frametitle{Problem \#2}

\begin{semiverbatim}
toPickle(...)(materialize[Person, Nothing])

                          \arrowdown

toPickle(...)(new Generic[Person, String :: HNil] \{ ... \})

\end{semiverbatim}

\begin{itemize}
\item Now the macro gets to finally expand
\item But unfortunately typer now expects \texttt{Generic[Person, Nothing]}
\item And that leads to the compilation error we observed
\end{itemize}
\end{frame}

\begin{frame}[fragile]
\frametitle{Let's take a step back}

\begin{semiverbatim}
implicit val personIso:
  Generic[Person, String :: HNil] = ...
implicit val repoIso:
  Generic[Repo, Url :: String :: HNil] = ...
implicit val commitIso:
  Generic[Commit, Id :: Array[Byte] :: HNil] = ...

implicit def genericPickler[T, R]
  (implicit iso: Generic[T, R],
            p: Pickler[R]): Pickler[T] = ...
toPickle(Person("Eugene"))

\end{semiverbatim}

\begin{itemize}
\item No macros for now. Can the compiler figure out this one?
\item Yes, it can, because we don't have conflicting implicit instances
\item Therefore all that we need here is just a little nudge from the macro
\end{itemize}
\end{frame}

\begin{frame}[fragile]
\frametitle{Fundep materialization}

\begin{itemize}
\item My first try was the \texttt{onInfer} callback (New Year's Eve 2013)
\item But later we found out a beautifully simple solution (May 2013)
\item Let macros expand even if they contain undertermined type params
\item The type of the expansion will help the compiler infer the undets!
\end{itemize}
\end{frame}

\begin{frame}[fragile]
\frametitle{Summary of generic programming techniques}

{\tabulinesep=1.5mm
\begin{tabu}{ | p{2cm} | p{3cm} | p{2.5cm} | p{2.5cm} | }
  \hline
  Technique & \parbox{3cm}{Client-side \\ convenience} & Performance & \parbox{2.5cm}{Library-side \\ convenience} \\ \hline
  Manual & Bad & Excellent & Excellent \\ \hline
  Reflection & Excellent & Bad & Decent \\ \hline
  Typelevel & Good / \text{\color{blue}{Excellent\textsuperscript{$\S$}}} & Good & Good \\ \hline
  Macros\textsuperscript{\textdagger} & Good / Excellent\textsuperscript{$\ddagger$} & Excellent & Decent \\
  \hline
\end{tabu}
}

\textsuperscript{\textdagger} Requires def macros (Scala 2.10.0+)\\
\textsuperscript{$\ddagger$} Requires implicit macros (Scala 2.10.2+)\\
\text{\color{blue}{\textsuperscript{$\S$} Requires fundep materialization (Paradise, maybe Scala 2.11.0+)}}
\end{frame}

\begin{frame}[fragile]
\frametitle{}

\vskip40pt
\begin{center}
\text{\color{blue}{\Large{Summary}}}
\end{center}
\end{frame}

\begin{frame}[fragile]
\frametitle{Summary}

\begin{itemize}
\item Macros can effectively abstract away boilerplate
\item In Scala 2.10 macros are useful
\item In Scala 2.11 macros will become enjoyable
\item Typelevel programming can sometimes be a viable alternative
\end{itemize}
\end{frame}

\begin{frame}[fragile]
\frametitle{Summary of generic programming techniques}

{\tabulinesep=1.5mm
\begin{tabu}{ | p{2cm} | p{3cm} | p{2.5cm} | p{2.5cm} | }
  \hline
  Technique & \parbox{3cm}{Client-side \\ convenience} & Performance & \parbox{2.5cm}{Library-side \\ convenience} \\ \hline
  Manual & Bad & Excellent & Excellent \\ \hline
  Reflection & Excellent & Bad & Decent \\ \hline
  Typelevel & Good / Excellent\textsuperscript{$\S$} & Good & Good \\ \hline
  Macros\textsuperscript{\textdagger} & Good / Excellent\textsuperscript{$\ddagger$} & Excellent & Decent \\
  \hline
\end{tabu}
}

\textsuperscript{\textdagger} Requires def macros (Scala 2.10.0+)\\
\textsuperscript{$\ddagger$} Requires implicit macros (Scala 2.10.2+)\\
\textsuperscript{$\S$} Requires fundep materialization (Paradise, maybe Scala 2.11.0+)
\end{frame}

% explain availability of all the features (including c.Tree)
% verify that all questions end with qmarks, mention that in the intro

\end{document}
